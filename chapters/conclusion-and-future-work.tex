\section{Limitations in Gephi}

As Gephi is still in the pre-1.0 stage, we encountered several
limitations using the tool, some of which forced us to implement
alternative or less than optimal approaches.

\subsection{No auto apply in the Partition Component}
No auto apply in the Partition Component. No Dynamic Collection data
structure.

\subsection{Playback issues in the Timeline component}

We also encountered some issues related to the playback of dynamic
graphs in Gephi. The playback is based on time intervals, where the
start and end bounds for the interval is encoded as a double. As you
play back the graph evolution, these bounds suffer from rounding errors,
which might lead to an interval that overlaps to simulation states.
These intervals are effectively filter, stripping the graph of nodes and
edges that do not belong in the defined interval. However, if the
interval crosses the boundary of two different simulation states, the
edges and nodes from both states will be included, which often leads to
a sudden increase in number of edges during the animation, which
abruptly disappear again. However, this issue is easily worked around by
defining a sufficiently short interval. We usually set the interval
bound to be of length 0.1, as this in our experience never leads to the
issues with the time interval overlapping different states. We confirmed
this by playing back the graph and controlling the start and end bounds
output on the Filter component, as well as visually confirming the
absence of any edge animation irregularities. Testing with the Cyclon
protocol, we were also able to visually confirm that the number of edges
remain consistent for each cycle in the context pane on the top right
side of the Gephi GUI\@.

\subsection{Dynamic Data Structures}

Gephi comes with its own set of data structures for representing
dynamic attributes on

Missing a DynamicCollection data structure Currently, only single value
data structures are represented.

\subsection{No Map Data Structure}
It would be useful to be able to store a map data structure on each
graph. If maps were supported, you could derive information such as how
many duplicate messages a particular node received on a specific topic
or single dissemination session. Also, it would enable us to see how
many publication messages a particular node sent or received for a
specific topic.

\subsection{No support for graph or workspace metadata}
Would be helpful to be able to list all topics on the graph and sort
then according to number of subscribers.

\subsection{Saving a Gephi session}
filters are not preserved.

\section{Conclusion}
\section{Future Work}
\subsection{Implementing a pub/sub plugin for Gephi}
\subsection{Improvements to existing solution}
