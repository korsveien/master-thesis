\section{Limitations in Gephi}

As Gephi is still in the pre-1.0 stage, we encountered several
limitations using the tool, some of which forced us to implement
alternative or less than optimal approaches.

\subsection{Playback issues in the Timeline component}

We encountered some issues related to the playback of dynamic
graphs in Gephi. The playback is based on time intervals, where the
start and end bounds for the interval is encoded as a double. As you
play back the graph evolution, these bounds suffer from rounding errors,
which might lead to an interval that overlaps to simulation states.
These intervals are effectively filter, stripping the graph of nodes and
edges that do not belong in the defined interval. However, if the
interval crosses the boundary of two different simulation states, the
edges and nodes from both states will be included, which often leads to
a sudden increase in number of edges during the animation, which
abruptly disappear again. However, this issue is easily worked around by
defining a sufficiently short interval. We usually set the interval
bound to be of length 0.1, as this in our experience never leads to the
issues with the time interval overlapping different states. We confirmed
this by playing back the graph and controlling the start and end bounds
output on the Filter component, as well as visually confirming the
absence of any edge animation irregularities. Testing with the Cyclon
protocol, we were also able to visually confirm that the number of edges
remain consistent for each cycle in the context pane on the top right
side of the Gephi GUI\@.

\section{Conclusion}

\section{Future Work}

\subsection{Visualizing with Custom Colors and Sizes}

Most of the visualizations in this thesis use text labels and color in
order to convey information regarding the performance of the pub/sub
system in question. However, it would also be interesting to visualize
metrics using shapes, sizes and custom colors. For example, edges could
be of different thickness, according to how many topic attributes were
added to them. Also, special nodes such as the \emph{rendezvous} nodes
in Scribe could have a custom color to them in order to identify them
easily. Alternatively, the nodes could be of a different sizes than
``regular'' nodes. This is different than the current way of applying
colors through the \emph{Statistics Component} in Gephi, as it is
dependent on the existence of a node attribute. Any custom color or size
would have to be applied by the Collector. However, due to a bug in the
software library used to build the \gexf{} files, this was not possible
to implement.

\subsection{Improving Report File Sizes}
Include a table of \gexf{} file sizes here.

\subsection{Including Associative Arrays in Gephi}

It would be useful to be able to store a map data structure on each
graph, i.e.\ associative arrays. If associative arrays were supported,
a user could derive information such as how many duplicate messages a
particular node received on a specific topic or single dissemination
session. Also, it would enable the user to see how many publication messages a
particular node sent or received for a specific topic. Implementing such
a structure will most likely involve modifying the source code of Gephi.
This is fully possible, since Gephi is an open source product and hosted
in a public repository.\footnote{\url{http://github.com/gephi/gephi}}
This would require quite an effort however, as the code base is quite
large and it would take time to gain the necessary insight into it.

\subsection{Implementing Global Attribute Visualization}

Currently, it is not possible to visualize graph attributes in Gephi, or
indeed list any global attribute in the graph. This would be a useful
feature to implement as a plugin. For example, it would be helpful to be
able to list all topics on the graph and sort then according to number
of subscribers. Our workaround for this is to apply a node label to all
nodes which describes a global attribute. Such global attributes include
hit-ratio and average control messages sent and received. Visualizing
these attributes on every node can be misleading, as it gives the
impression that the values are unique to every node. It would be better
to display these values in a separate panel, or perhaps even as a large
floating text label in the Graph view.

\subsection{\demo{} as an Interactive Monitoring System}

As mentioned, the visualization of pub/sub systems are done offline,
after the execution of the pub/sub system is finished. It is not
intended to be a real-time system, where data is pulled and metrics are
calculated during system execution. When inspecting the overlay using
the \emph{Visualization Unit}, users may delete nodes in order to
determine the effect it has on the overlay topology by recalculation
such metrics using the \emph{Statistics Component} in Gephi. However, it will
not affect any custom metrics which are calculated by the Collector such
as control messages sent and received as these metrics are calculated
after the execution of the pub/sub system. In order to affect such
metrics, the architecture would have to be changed in order to
accommodate for a two-way communication, where the Collector would be
able to issue commands to the pub/sub nodes. Also, the Collector would
have to collect data, calculate metrics and stream this information to
Gephi in real-time. Gephi does indeed include an API for streaming data,
so it is possible in theory, although in practice there are some issues
with this API.\ This would be incredibly useful however, both as a
monitoring system and as an environment for experimenting with protocols
and their behavior. For example, imagine being able to right-click a
node in Gephi, and ask this node leave the network. The Collector would
then issue a leave command to the node in question, and the user could
then potentially observe what happens in the Visualization Unit. The
interactivity of such a feature would be incredibly engaging and useful
both for developers and researchers, as well as students. Such
a feature requires extensive re-engineering of \demo{}, but would be
well worth the effort.
