\section{Overview}

In topic-based pub/sub, the published events are expressed as discrete
\emph{topics}. This variation of pub/sub benefits from the large amount
of research in group communication, as subscribing to Topic \emph{t} can
be viewed as joining Group \emph{t}~\cite{Eugster:2003}. This makes
topic-based pub/sub a promising approach for efficient message
dissemination in P2P environments, and an increasing number of such
systems are being deployed~\cite{Triantafillou:2009}.

\section{Desired System Properties}
In order to provide correct and efficient delivery of notifications in a
decentralised social network, a high number of system properties are
deemed desirable~\cite{Setty:2012}. More specifically, these challenges
include:

\subsection{Correct delivery}
All notifications should be delivered to the
correct recipient. Both false positives and false negatives should be
avoided.

\subsection{High hit-ratio}
Notifications are delivered to a very high
percentage of subscribers in the presence of churn and all subscribers
when no churn is present. This is similar to correct delivery except for
not taking false positives into account.

\subsection{Fast recovery}
The overlay should quickly recover from a
period of churn.

\subsection{Low average node degree}
The overlay nodes should have a low
node degree as possible to achieve scalability with regards to number of
topics.

\subsection{Topic connectivity}
The routing of an event only includes the
subscribers who registered their interest for the topic (i.e.\
relay-free routing).

\subsection{Scalability}
The system should scale in terms of number of
topics, number of nodes, number of topics a node is interested in and
number of nodes interested in a topic.

\subsection{Efficient dissemination}
Event dissemination should have a low
delay with little duplicate delivery, and the load of routing messages
should be distributed fairly.

\subsection{Low overlay maintenance cost}
Managing the overlay topology
should be as inexpensive as possible. Maintenance might include mending
dissemination structures such as multicast trees when nodes fail, but
also how to include joining nodes in the structure and allowing nodes to
leave gracefully.

Designing a system with all of these properties
presents a challenge, as several of the desired characteristics are
fundamentally at odds with each other. Maintaining a low node degree
makes it difficult to maintain \emph{topic connectivity}, while
avoiding duplicate message delivery conflicts with being robust in the
presence of churn. There is also a trade-off in robustness and
reliability depending on the approach taken to disseminating messages.
Specialised overlays that build dissemination structures such as
multicast trees provide fast and reliable message delivery with no
duplication of messages. However, they are fragile and susceptible to
churn. Epidemic dissemination on the other hand is more robust, but does
not provide full reliability as it lacks deterministic delivery
guarantees.

There is also a trade-off between the navigability of the overlay and
the message overhead. Stanley Milgram famously demonstrated the
\emph{small-world phenomena} in~\cite{milgram1967small} where he showed
that any two participants in a network was likely to be connected
through a low number of intermediaries. Taking this phenomena into
consideration has proved to be a useful approach when constructing
decentralised overlays, as they provide a highly navigable network due
to the small average shortest path length. A popular approach is to
create one or more long jump links between nodes to provide better
routing capabilities. More specifically, these links are usually created
by utilising a distance function in the name space, where the
probability of creating a link increases with the distance between them.
The subscription interest of such nodes are usually not taken into
consideration when creating such links. Consequently, the message
overhead in the system is increased as more relays are introduced in the
overlay.

Many existing systems suffer from shortcomings that originate wanting
to include a high number of desired properties described above. This
motivates further research into these systems and how they compare in
terms of promoting these desirable properties.

\section{Handling Trade-offs}

Designers of existing approaches have been facing the challenges of
handling the trade-offs discussed in the previous section. One of these
challenges is building both a reliable and robust overlay. A naive
approach to this problem would be to create a separate overlay for each
topic as in TERA~\cite{Baldoni:2007}. However, this approach suffers from
poor scalability as the number of nodes and topics increase. Another
approach would be to structure the overlay by creating a spanning tree
per topic as seen in Scribe~\cite{Castro:2002}, Bayeux~\cite{Zhuang:2001}
and Magnet~\cite{Girdzijauskas:2010}. However, these structures are
conceptually fragile in the presence of churn, requiring mechanisms for
mending the structure when nodes fail. This increases the overhead of overlay
maintenance.  Also, the root node of the spanning tree represents a
single point of failure as well as a bottleneck in the system. This is
especially true for popular topics where all events must travel through
the root node. In Scribe, the root node is used as a \emph{rendezvous}
point for topics by using the routing capabilities of the underlying
Pastry~\cite{Rowstron:2001} DHT\. Such dedicated nodes represent a
single point of failure in addition to being detrimental to load
balancing and scalability.

Minimising the average node degree while simultaneously achieving a
\emph{topic-connected} overlay is another difficult trade-off to
consider. \emph{Topic-connectivity} is achieved when no other than the
nodes who registered their interest in a topic takes part in routing
events for that topic. The desired goal with regards to
topic-connectivity is not only to avoid routing events through
uninterested nodes, but also to minimise the node degree by reusing
links for several topics. This approach achieves better scalability with
regards to the number of topics in the system.  Also, it decreases the
message overhead incurred by both event dissemination and overlay
maintenance mechanisms such as heartbeat messages. In addition to this,
keeping an overlay topic-connected simplifies the message routing
mechanism as no designated relay or gateway node needs to be implemented
in the protocol such as in Scribe and Vitis.

ElastO~\cite{Chen:2013} propose an interesting approach to overlay construction,
that aims at constructing a topic-connected overlay (hereby referred to
as a TCO) while maintaining a
low node degree. The construction of the overlay is performed by a centralised
component which requires global knowledge, and by using any existing static or
decentralised algorithm. However, the maintenance of the overlay is
performed in a distributed manner in response to churn events. By using
a centralised algorithm for overlay construction, ElastO can provide a
more optimal TCO than decentralised solutions, while still maintaining
the high performance of a decentralised repair mechanism to handle node
departure or arrival.

With regards to the reuse of links for several topics, an observed
correlation~\cite{Liu:2005} between subscription sets in practical
workloads is useful to consider when constructing overlays. This
observation is exploited in Poldercast in order to lower the number of
links to maintain. Also, it is used as a basis for overlay construction
in both StaN~\cite{Matos:2010} and SpiderCast~\cite{Chockler:2007}.
However, these two protocols only provide a probabilistic guarantee that
the resulting overlay will be fully topic-connected. In contrast,
PolderCast claim deterministic guarantees of providing a TCO\. However,
this relies on two factors: (1) that there is no churn in the system,
and (2) that the underlying Cyclon~\cite{Voulgaris:2005} protocol, which
is used for peer sampling in PolderCast, can guarantee a connected
overlay.  Consequently, the deterministic guarantees of PolderCast could
be questioned.\ daMulticast~\cite{Baehni:2004} on the other hand provide
a deterministic guarantee of topic-connectivity through quite a
different approach of overlay construction. More specifically,
daMulticast constructs a topic hierarchy, where events are disseminated
through gossiping each level of this hierarchy in a bottom-up approach.

As mentioned in section 2, several protocols attempts to create an overlay that
exhibits \emph{small-world properties}. In Vitis, the subcluster
together with the relay paths form an overlay similar to a small-world
networks, which benefits the routing delay but includes
uninterested gateway and relay nodes. In Magnet, small-world properties
are provided by the underlying Oscar DHT~\cite{girdzijauskas2007oscar}
which also cluster similar nodes together.

\subsection{Overlay construction}

There are several different approaches to overlay construction.
Structured approaches such as dissemination trees have already been
mentioned, but there are also unstructured approaches. In Quasar
\cite{Wong:2008}, a novel approach to event dissemination using random
walks removes the need for a structured overlay.  There are also hybrid
approaches to structuring overlays such as in ElastO~\cite{Chen:2013}
and Vitis~\cite{Rahimian:2011}. In ElastO, the bootstrapping of the
graph is performed by a centralised entity, while in Vitis nodes with
similar interests are clustered together. A topic in Vitis might consist
of several subclusters in Vitis, which are connected to each other
through relay paths. This creates an overlay that is similar to
dissemination trees, but where single nodes have been replaced with
clusters of nodes. However, the drawbacks are still similar to the ones
found in systems relying on multicast trees, as it relies on designated
gateway nodes within subclusters communicating with rendezvous nodes
along the relay path. In Poldercast~\cite{Setty:2012}, a structured ring
per topic is used in combination with a form of epidemic dissemination
that resembles gossiping. Publishers are themselves part of the ring of
the topic they publish, and the structures are combined into a single
overlay through random links. Such an hybrid approach is an attempt at
balancing the reliability of a structured overlay with the robustness of
epidemic dissemination. When it comes to node degree however, Poldercast
might introduce hotspots in the system as the distribution of random
links might be skewed.

In Magnet, the aforementioned subscription correlation is used to build
dissemination trees such as the ones seen in Scribe and Bayeux. As
mentioned these structures are not ideal in dynamic systems as they
require maintenance. However, tree structures do have an advantage when
it comes to the case of communication overhead, as they naturally avoid
any duplicate messages. This is not the case for dissemination protocols
who rely on epidemic dissemination such as in daMulticast and
PolderCast.  However, the duplicate message delivery is also the reason
why such epidemic dissemination techniques are so robust during churn.
Furthermore, there is usually an adjustable fanout parameter in such
systems which can be manipulated in order to control the number of
messages that are forwarded by a node. Thus, there is some control over
the amount of communication overhead in the system. Both PolderCast and
daMulticast include such a fanout parameter. Also, it bears mentioning
that structured overlays are not immune to communication overhead, as
the structures usually requires both maintenance and mending in case of
failure. Control messages such as heartbeats are commonplace in such
protocols e.g.\ in Scribe where the each non-leaf node in the multicast
sends heartbeats to its children periodically.  This increases bandwidth
consumption and adds a higher communication overhead compared to
unstructured overlays where such maintenance is not required.

Node degree is another important issue to consider when designing
overlays. A low average node degree increases scalability as topics and
number of nodes in the system increases. In protocols who rely on an
underlying DHT, node degree is usually either constant or a logarithmic function of the
total number of nodes in the graph. Such is the case in Bayeux which
relies on Tapestry~\cite{tapestry}, or Scribe which relies on
Pastry~\cite{Rowstron:2001}. Other implementations might have a constant
node degree, which is the case in Vitis which might result in the
separation of a topic into subclusters. Some extreme examples include
TERA~\cite{Baldoni:2007} and daMulticast who have a node degree that
grows in the order of the number of topics the node has subscribed to.
In the worst
case scenario, this is also the case in PolderCast, as maintaining a low node
degree depends on the degree of correlation in the subscription sets.
Indeed, when using workloads from Facebook, the node degree in
PolderCast grows almost linearly with subscription size, as shown in
\cite{Setty:2012}. This suggests a scalability issue in a scenario where the
subscription correlation is weak.

%{../tables/comp-overlay} <- vim gf
\begin{table}
\centering
\resizebox{\columnwidth}{!}{%
\begin{tabular}{ccccccc}
    \toprule
    Protocol                         & Overlay      & Structures? & TCO?\    & Central nodes* & sub.\ corr.? & Node degree \\
    \midrule
    Scribe~\cite{Castro:2002}        & Structured   & Trees       & \xmark{} & RV             & \xmark{}     & $O(\log|\mathcal{V}|)$ \\
    Magnet~\cite{Girdzijauskas:2010} & Structured   & Trees       & \xmark{} & Relays         & \cmark{}     & $O(1)$\\
    Bayeux~\cite{Zhuang:2001}        & Structured   & Trees       & \xmark{} & RV             & \xmark{}     & $O(\log|\mathcal{V}|)$\\
    Vitis~\cite{Rahimian:2011}       & Hybrid       & Trees       & \xmark{} & RV\ampers{}GW  & \xmark{}     & $O(1)$\\
    StaN~\cite{Matos:2010}           & Unstructured & None        & prob.    & None           & \cmark{}     & $O({|\mathcal{T}_v}|)$\\
    SpiderCast~\cite{Chockler:2007}  & Unstructured & None        & prob.    & WB             & \cmark{}     & $O(K\cdot(|\mathcal{T}_v|))$\\
    daMulticast~\cite{Baehni:2004}   & Unstructured & None        & det.     & None           & \xmark{}     & $\Theta({|\mathcal{T}_v}|)$\\
    Quasar~\cite{Wong:2008}          & Unstructured & None        & \xmark{} & None           & \xmark{}     & Unknown\\
    PolderCast~\cite{Setty:2012}     & Hybrid       & Rings       & det.     & None           & \cmark{}     & $O({|\mathcal{T}_v}|)$\\
    ElastO~\cite{Chen:2013}          & Structured   & Ring        & det.     & None           & \xmark{}     & $O({\rho \log |\mathcal{V}}| |\mathcal{T}|)$\\
    \bottomrule
    \multicolumn{5}{l}{*RV:\ Rendezvous GW:\ Gateway WB:\ Weak bridge}\\
\end{tabular}
}%
\caption{Comparison of the different protocols and their overlay properties}
\label{table:comp-overlay}
\end{table}


Table~\ref{table:comp-overlay} provides an overview over several
different state-of-the-art protocols, comparing their different system
properties such as topic connectivity and whether or not it takes
advantage of the observed subscription correlation.  Note that
PolderCast has received benefit of the doubt in this table, and have
been marked as providing a deterministic guarantee of
topic-connectivity. Magnet has a different approach to the spanning tree
structures, where messages are disseminated bottom-up. This means that
the root node is not a rendezvous node according to  the traditional
definition~\cite{baldoni2005distributed}, but it is still conceptually
a single point of failure as it is responsible for propagating messages
back down the tree when it receives a message. In SpiderCast, there is a possibility
of the overlay forming into a pattern of highly connected clusters
inter-connected through a small number of links which we refer to as
\emph{weak bridges}. The node degree in SpiderCast also relies on the
\emph{K-coverage} parameter of the system, where, for each topic, a node
attempts to connect to $K$ neighbours who share the same interest.
Protocols who rely on a underlying DHT typically have a node degree
which grows logarithmically with the number of nodes in the system. The
exception is Magnet which leverages a DHT providing small-world
properties~\cite{girdzijauskas2007oscar}, creating a fixed node degree
that is independent of both subscription size and number of topics
\cite{Zhuang:2001}.  Note that the node degree in Quasar is omitted from
the table, as it is dependent on the implementation of the bloom filters
used to represent the neighbours. In ElastO, $\rho$ is a system parameter
which balances between average and maximum node degrees when choosing
new edges to recover from churn. Note that even though ElastO has a
higher node degree than the centralised solutions, it provides a more
optimal TCO.\

\section{Event dissemination}

In terms of event dissemination, it should be mentioned that some of
the systems described earlier do not concern themselves with this aspect, and
focus instead of the construction and maintenance of the overlay
itself. In specific, this includes SpiderCast, StaN and ElastO. Thus, any
discussion regarding dissemination technique or routing performance
will be irrelevant for these systems. For other systems however, a
comparison of techniques is in order.

As mentioned, the dissemination in systems relying on multicast trees
removes any message duplication and usually offers on average
dissemination of events in logarithmic time as is the case in Scribe,
Magnet and Bayeux. In Vitis, event dissemination is performed by
flooding inside the subcluster, while simultaneously forwarding the
event to other subclusters if needed. As mentioned, gossiping is the
main approach in both daMulticast and PolderCast.  Gossiping usually
implies an exponential dissemination speed, however, there might be
other implementation specific factors in play which inhibits this
property of gossiping. As an example, in PolderCast, skewed random link
distribution might be detrimental to the speed of the gossiping
protocol. Quasar~\cite{Wong:2008} proposes quite a different approach
to event dissemination, as routing is performed by having nodes install
routing vectors in nearby overlay neighbours. Messages are disseminated
through random walks, which are directed towards the subscribers when
passing through a node with the relevant routing information. This
approach is likely to be highly robust against churn. However, as
observed in~\cite{Wong:2008} the hit ratio stagnates at 97\% in a
static system. This is due to a phenomenon where some group members
might be obscured by other members who absorb messages
from all directions. It could be interesting to see how this phenomenon
could affect the system during churn.

%{../tables/comp-dissemination} <- vim gf
\begin{table}
\centering
\resizebox{\columnwidth}{!}{%
\begin{tabular}{ccccc}

\toprule
Protocol                         & High hit-ratio during churn & 100\% hit-ratio in absence of churn & Message Delay              & Avg.  Duplication Factor \\
\midrule
Scribe~\cite{Castro:2002}        & \xmark{}                    & \cmark{}                            & $O(\log|\mathcal{V}|)$     & None\\
Magnet~\cite{Girdzijauskas:2010} & Unknown                     & Unknown                             & $O(\log|\mathcal{V}|)$     & None\\
Bayeux~\cite{Zhuang:2001}        & Unknown                     & Unknown                             & $O(\log|\mathcal{V}|)$     & None\\
Vitis~\cite{Rahimian:2011}       & \cmark{}                    & \cmark{}                            & $O(\log^2|\mathcal{V}|/k)$ & Scoped flooding\\
daMulticast~\cite{Baehni:2004}   & \cmark{}                    & \xmark{}                            & $O(\log|\mathcal{V}|)$     & Gossiping\\
Quasar~\cite{Wong:2008}          & \cmark{}                    & \xmark{}                            & Unknown                    & Random Walk \\
PolderCast~\cite{Setty:2012}     & \cmark{}                    & \cmark{}                            & $O(\log|\mathcal{V}_t|)$   & $\leq fanout(f)$\\
\bottomrule

\multicolumn{5}{l}{*RV:\ Rendezvous. GW:\ Gateway. WB:\ Weak bridge.}\\
\end{tabular}
}
\caption{Comparison of the different protocols and their routing properties}
\label{table:comp-dissemination}
\end{table}


Table~\ref{table:comp-dissemination} describes the routing properties of
different protocol being discussed. Note that protocols relying on a DHT
usually have an expected delay which is logarithmic or squared
logarithmic with the total number of nodes in the system. In Vitis, the
underlying DHT provide squared logarithmic routing complexity with the
total number of nodes divided by $k$, the number of long-range
neighbours. To the best of our knowledge, there is no evaluation of the
hit-ratio of Magnet or Bayeux, which is the reason of these being marked
as unknown. The message delay of the system is a description of the
expected path length of a event before it reaches a subscriber in number
of hops. Systems relying on an underlying DHT usually provides routing
performance which is logarithmic to the number of nodes in the system.
Magnet differs in its approach to routing, as it relies on random walks
with associated TTL values. However, as described in
\cite{Girdzijauskas:2010} this value is usually set to the logarithm of the
total number of nodes in the system. The average duplication factor
describes the message overhead of the system, where gossiping is usually
dependent on a fanout system parameter. The novel approach in Quasar
means message overhead is dependent on the number of parallel random
walks initiated by a node. As described earlier, protocols who create
specialised dissemination structures, in these cases spanning trees,
have no message duplication.
