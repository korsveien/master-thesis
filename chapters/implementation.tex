\section{Implementation Strategy}

\subsection{Traditional Approach to Evaluating P2P Overlays using PeerNet}

\subsection{Our Approach}

\subsection{Using Test-Driven Development}

Software Development Methodology is an active area of research
which is in part driven by the business needs of the private
sector\cite{janzen2005test}. One popular practice is so-called Test-Driven
Development (TDD). The promoters of TDD claims it increases
productivity and reduces the number of bugs and defects in the
code significantly~\cite{beck2003test}. Research
efforts performed at IBM~\cite{maximilien2003assessing} seems to
lend credibility to these claims. However, the use of TDD is not
prevalent in academia, and in~\cite{janzen2005test} they
recommend further research into the field in order to better
determine its effects.

Using TDD means writing tests before writing any code. There are
different types of test. \emph{Unit Tests} targets small,
independent pieces of code, typically methods within a single
module or component, while \emph{Integration Tests} aim to test
code across such modules and components in order to determine
how well they integrate with each other. In our work, we only
took advantage of Unit Tests where suitable using the
JUnit~\cite{junit} and Mockito~\cite{mockito} libraries.
We could also have benefited from a suite of integration tests,
as our implementation is heavily dependent on interoperating
components, as well as file and network IO\@. However, writing
these sort of tests would simply be too time consuming compared
to writing smaller unit tests.

The TDD approach to software development is best described through the
Red-Green-Refactor mantra, which is a central part of the
TDD-philosophy. It can be described through the following steps:

\begin{description}
    \item[Step 1:] Write a test that fails. (Red)
    \item[Step 2:] Make the test pass. (Green)
    \item[Step 3:] Refactor the code while making sure the test
        still passes. (Refactor)
\end{description}

In our experience this routine has been helpful when working
with our implementation code, as it enables us as developer to
refactor with confidence achieving more maintainable code and a
more thoughtful software design. Since we share our
implementation code with the research community by hosting it in
a open repository, any tool or method that helps us improve the
design and maintainability of our project is of great value to
us. Using TDD forced us to think more deeply about what
functionality to implement and how to structure and split the
problem domain into smaller function points. We believe that in
the end, following TDD where its suitable is beneficial to both
programmer productivity as well as programmer happiness. Also,
we are confident that this practice decreased the amount of
technical debt in our project, a problem we find to be commonplace in academia.

\subsection{Aggregating data}
Using the coordinator as a data aggregate. Aggregated data
stored in gexf format.

\section{Implementing Protocols in PeerNet}
\subsection{Updating Existing Protocols}
\subsection{Protocols}
\subsection{Observers}

\section{Extending PeerNet}

\section{Implementation Challenges}

\subsection{Refactoring for distributed simulations}
It is often the case that taking something that works well
locally and transparently distributing it leads to difficulties.

\subsubsection{Messages too large in PolderCast}
\subsubsection{Intializing topics}

\subsection{Handling Memory Leaks in Gephi Toolkit}
During our work we encountered a particularly nasty bug in the Gephi
Toolkit. When processing GEXF files, it is necessary to create a Gephi
workspace, do the file processing, and then close this workspace. Each
time a new workspace is created, five new threads are spawned in the
JVM\@. However, when closing the workspace, these threads are not
terminated. This is a memory leak, as the threads allocate space on the
Heap for every thread created. As we often have to traverse through
folders when processing these files in a for-loop, creating and closing
a new workspace for each iteration, this inevitably leads to a
\texttt{OutOfMemoryException} and the JVM process exiting
unsuccessfully.

The snapshot version of the toolkit included a fix for this issue,
however this version was too unstable and broke backwards compatibility
which affected our implementation code. We implemented a workaround
which involved serializing the reports received by the collector into
JSON strings using the Google GSON~\cite{gson} library. The collector
would then spawn a new JVM process, passing this string to it.  This is
far from an elegant solution, but works as a temporary fix until there
is a newer stable release of the toolkit that includes the fix to this
issue.


\subsection{GEXF file sizes}
The GEXF format may not be scalable. However, gephi accepts
archive files. Zipping gexf files reduced file sizes up to 93%.

\section{Limitations in Gephi}

As Gephi is still in the pre-1.0 stage, we encountered several
limitations using the tool, some of which forced us to implement
alternative or less than optimal approaches.

\subsection{No auto apply in the Partition Component}
No auto apply in the Partition Component. No Dynamic Collection data
structure.

\subsection{Playback issues in the Timeline component}

We also encountered some issues related to the playback of dynamic
graphs in Gephi. The playback is based on time intervals, where the
start and end bounds for the interval is encoded as a double. As you
play back the graph evolution, these bounds suffer from rounding errors,
which might lead to an interval that overlaps to simulation states.
These intervals are effectively filter, stripping the graph of nodes and
edges that do not belong in the defined interval. However, if the
interval crosses the boundary of two different simulation states, the
edges and nodes from both states will be included, which often leads to
a sudden increase in number of edges during the animation, which
abruptly disappear again. However, this issue is easily worked around by
defining a sufficiently short interval. We usually set the interval
bound to be of length 0.1, as this in our experience never leads to the
issues with the time interval overlapping different states. We confirmed
this by playing back the graph and controlling the start and end bounds
output on the Filter component, as well as visually confirming the
absence of any edge animation irregularities. Testing with the Cyclon
protocol, we were also able to visually confirm that the number of edges
remain consistent for each cycle in the context pane on the top right
side of the Gephi GUI\@.


\subsection{Dynamic Data Structures}

Gephi comes with its own set of data structures for representing
dynamic attributes on

Missing a DynamicCollection data structure Currently, only single value
data structures are represented.

\subsection{No support for graph or workspace metadata}
Would be helpful to be able to list all topics on the graph and sort
then according to number of subscribers.

\subsection{Saving a Gephi session}
filters are not preserved.

