\section{Intended Audience}

The main target audience for this master thesis are researchers within
the field of distributed systems, specifically those whose field of
study is peer-to-peer (P2P) decentralised networks and the
Publish-Subscribe paradigm. But any computer science student with a
healthy interest in the field should be able to both understand and
benefit from reading this thesis.


\section{Motivation for Thesis}

The Publish-Subscribe (pub/sub) communication paradigm is receiving an
increasing amount of attention from the research community, as it
provides a loosely coupled and scalable interaction scheme suitable for
large-scale distributed systems~\cite{Eugster:2003}. It has also shown
to be a useful approach for several business applications such as
financial data dissemination~\cite{tibcorv} and application
integration~\cite{goops}.  Topic-based pub/sub has also seen more novel
applications such as decentralised social networks. More specifically,
the issue of delivering notifications in such a network is a task
especially suited for this approach. This motivates further
investigation into such systems, comparing their performance and
analysing their characteristics and shortcomings.

In~\cite{Setty:2012}, we provided a mini-survey of several existing
state-of-the-art topic-based pub/sub systems, illustrating their
different characteristics and highlighting their shortcomings. However,
there is a need for an ever deeper evaluation and comparisons of these
systems, and we feel such a comparison would be a great contribution to
the research community.

\section{Problem Statement}

\section{Chapter Overview}

\begin{description}
    \item[Chapter~\ref{ch:background}] \hfill \\
        Provides background for pub/sub systems, p2p, overlays, social
        networks, gephi, peernet, poldercast and scribe
    \item[Chapter~\ref{ch:design-challenges}] \hfill \\
        Desired System Properties and Trade-offs. Taken from essay.
    \item[Chapter~\ref{ch:vizpub}] \hfill \\
        Describe vizpub, expand what is written in the demo paper.
        Provide use case examples and experiences using vizpub as a tool
        when updating poldercast. Emphasize uses in both developing,
        analyzing pub/sub systems as well as a tool for educational
        purposes. Also provide examples of visualizations, in particular
        attempt to visualize some of the evaluations performed in the
        poldercast paper, and see what additional information we can
        learn from it.
    \item[Chapter~\ref{ch:evaluation}] \hfill \\
        Use vizpub as a tool for evaluating experiments. We update both
        poldercast and scribe, in order to expand the evaluation
        performed in the poldercast paper. Emphasize how the vizpub tool
        along with the gephi framework enable us to easily expand the
        evaluation on particular metrics, as these are inlcuded in
        gephi. These metrics include, in-degree, out-degree, clustering
        coefficient and centralities.
    \item[Chapter~\ref{ch:conclusion-and-future-work}] \hfill \\
        Emphasize how future evaluations could be expanded by
        implementing statistics plugins in gephi, as well as extending
        the reporter interface. This enables researcher to more easily
        share their work, no more one-off code! Also, a vizpub plugin
        for gephi might be a good idea. This could provide a panel in
        gephi which could display global metrics such as hit-ratio,
        instead of displaying them as a label on every node. Also, this
        plugin could display the current reporter interval being
        visualized, as this is hard to derive from the timeline with
        several hundred intervals.
\end{description}
