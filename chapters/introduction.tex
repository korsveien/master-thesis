The Publish-Subscribe (pub/sub) communication paradigm is receiving an
increasing amount of attention from the research community, as it
provides a loosely coupled and scalable interaction scheme suitable for
large-scale distributed systems~\cite{Eugster:2003}. It has also shown
to be a useful approach for several business applications such as
financial data dissemination~\cite{tibcorv} and application
integration~\cite{goops}. Pub/sub systems have a large number of
important performance metrics related to delivery latency, bandwidth
consumption, communication overhead as well as computation overhead.
Many of these aspects of pub/sub systems are known to be at stake with
each other~\cite{Kermarrec:2013, Setty:2012}, forcing researchers to face difficult design
decisions when developing such systems in order to achieve a suitable
balance in performance.

Overlays play a key role in the pub/sub system architectures, where many
considerations need to be taken both with regards to overlay maintenance
as well as structural overlay properties such as clustering and
connectivity. These properties further add to the set of performance
metrics which must be studied and weighed against each other by
researchers and developers of pub/sub systems.

In this thesis we present \demo{}, a tool we propose for visualizing the
execution of any given pub/sub system step-by-step with respect to a set
of performance metrics. Our tool enables logging local information at
every node in the system under analysis, which is then collected at a
single site for analysis. When all data has been collected, the tool
will calculate global metrics at a system-wide scale. The resulting data
is stored to disk in a single file using the
\gexf{}~\footnote{GEXF (Graph Exchange XML Format) is a language for
    describing complex networks: \url{http://gexf.net/format/}} file
format. This file is then interpreted by the Gephi Open Graph Viz
tool~\cite{ICWSM09154} which enables the user to replay system execution
as well as visualize overlay structure, dissemination paths of
publications messages and various performance-critical metrics.

To the best of our knowledge, \demo{} is the first tool of its kind. And
although our main focus lies in the performance of pub/sub
systems, the tool is generic in design, and can be used in order to
visualize other types of overlay-based systems such as~\cite{Loukos:2014}.

The major benefit of our tool lies in giving both researchers, developers as
well as students the ability to gain a deeper understanding of the
properties of overlay-based systems, both in terms of their structure
and their dissemination schemes. The tool is capable of providing the
user with two different types of visualizations: (1) visualization of
overlay structure and (2) visualization of publication message
dissemination. Both types of visualization provides the user with the
ability to replay the system execution as well as pausing and analyzing the overlay at
different points in time. The visualizations mentioned above grants the user with an
insight into the system which is helpful in order to determine protocol
behaviour as well as potential weaknesses and anomalies of any given
pub/sub protocol. In this thesis, we describe some of the practical experiences we had
with the tool, which should serve as good examples of it utility.

Other benefits include the possibility of comparing different pub/sub
systems visually. The user can run different systems with fixed
parameters, driving the same workloads for publications and
subscriptions, and then compare the performance and characteristics of
the different systems at selected points in time. In
Chapter~\ref{ch:vizpub} we include such a comparison of two different
pub/sub systems.

As most of the interesting performance metrics across different pub/sub
systems remains the same, we specify a generic list of performance
metrics as well as a generic \emph{reporter interface}. This interface
is designed to provide the necessary data required to calculate these
metrics and constitute the only part of the architecture which is system
specific. This is an important point. The only work required from a
researcher or developer who whish to take advantage of our tool is to
implement this interface. The architecture of \demo{} is designed to be
modular and highly decoupled. The only contact point between the
running system and \demo{} is the reporter interface.

We also take advantage of the capabilities of \demo{} in order to extend
the evaluation performed in~\cite{Setty:2012} on a specific set of
structural overlay metrics. The advantage of using \demo{} for this
purpose lies in the fact that the Gephi framework grants us with
several algorithms for calculating certain metrics for ``free''. Also, adding
support for metrics which are not included in Gephi can be done by
developing plugins for Gephi which can be distributed via its online
marketplace~\footnote{Plugins for Gephi are available at the official
    Plugin Marketplace: \url{https://marketplace.gephi.org/}}. This encourages sharing code between
researchers, avoiding the individual researcher having to implement algorithms from scratch, which
we believe would be of great benefit to the research community.

We presented a poster and held a live demo of \demo{} at the ACM
International Conference of Distributed Event Based Systems (DEBS),
hosted in Mumbai in May 2014. The positive feedback from the researchers and
students attending the conference reassured us that there is a
demand for such a tool and that its potential usefulness is widely
understood and appreciated. The value of our contribution was further
demonstrated, as \demo{} received the prize for best poster and
demonstration. This reassures us that there is a need for such a tool
and that \demo{} may serve as a lasting contribution to the research
community at large. In order to encourage contributions and further
development of \demo{}, we host our implementation code in a public
repository~\footnote{\demo{} is hosted in a public repository: \url{http://github.com/vizpub/vizpub}}.

\section{Chapter Overview}

The following is a brief overview of the chapters included in this
thesis:

\begin{description}
    \item[Chapter~\ref{ch:background}] \hfill \\

        In this chapter, we explain key concepts and describe the tools
        and technologies we use in the work presented in this thesis.
        This chapter is meant to serve as a brief overview of relevant
        topics and should prove useful in order to fully understand and
        appreciate the work presented in this thesis.

    \item[Chapter~\ref{ch:design-challenges}] \hfill \\

        We extend the mini-survey found in~\cite{Setty:2012} with a set
        of additional pub/sub protocols. This chapter should be helpful
        in order to understand the different challenges researchers face
        when designing such systems. In particular it focuses on what
        trade-offs must be considered, and what are the advantages and
        drawbacks of certain design decisions.

    \item[Chapter~\ref{ch:vizpub}] \hfill \\

        We describe \demo, a tool we propose for visualizing the
        performance of overlay-based pub/sub systems. The chapter
        includes descriptions of the system architecture, as well as
        examples of visualizations. Also, we describe our experiences
        using the tool in order to further demonstrate its benefits both
        for researchers, developers as well as students.

    \item[Chapter~\ref{ch:evaluation}] \hfill \\

       We use \demo~in order to expand the evaluation performed
       in~\cite{Setty:2012} on a set of particular topology metrics. Our
       tool enable us to easily add these metrics to the evaluation, as
       the algorithm for calculating these properties of the overlay are
       included in the Gephi framework.

        % Use vizpub as a tool for evaluating experiments. We update both
        % poldercast and scribe, in order to expand the evaluation
        % performed in the poldercast paper. Emphasize how the vizpub tool
        % along with the gephi framework enable us to easily expand the
        % evaluation on particular metrics, as these are inlcuded in
        % gephi. These metrics include, in-degree, out-degree, clustering
        % coefficient and centralities.

    \item[Chapter~\ref{ch:conclusion-and-future-work}] \hfill \\

        The final chapter summarizes the contributions of this thesis.
        We discuss our results, as well as the future of \demo. We also
        discuss what are the major challenges moving forward, and how
        the tool could be expanded. It is our hope that this tool will
        prove useful for the research community in the future.

        % Emphasize how future evaluations could be expanded by
        % implementing statistics plugins in gephi, as well as extending
        % the reporter interface. This enables researcher to more easily
        % share their work, no more one-off code! Also, a vizpub plugin
        % for gephi might be a good idea. This could provide a panel in
        % gephi which could display global metrics such as hit-ratio,
        % instead of displaying them as a label on every node. Also, this
        % plugin could display the current reporter interval being
        % visualized, as this is hard to derive from the timeline with
        % several hundred intervals.

\end{description}
