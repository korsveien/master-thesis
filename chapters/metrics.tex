\section{Overview}

     There are several metrics that should be considered when evaluating the
     performance of these systems. This section will provide an overview of
     different metrics and their relevance to event dissemination in a P2P
     topic-based pub/sub systems. Discussing these metrics will be
     important, as they form one of the main contributions of this
     thesis. In general, a distinction can be made
     between evaluating the overlay itself or evaluating the efficiency of
     event dissemination i.e.\ the performance of message routing.

\section{Overlay metrics}
    When evaluating the overlay,
    the following metrics are relevant:

    \begin{description}

    \item[Average node degree]\hfill\\
        It is essential to understand how the system scales with regard
        to the number of topics a node is interested in. For example, if the number of
        edges increases linearly with the subscription size of a node,
        scalability suffers.

    \item[Maximum node degree]\hfill\\
        Also it might be interesting to measure the maximum node
        degree. As an example, having a relatively low average node
        degree while still having a rather high maximum node degree
        might reveal an unbalanced distribution of node degree in the
        graph.

    \end{description}

    The two metrics above could also be separated into measuring both
    in-degree and out-degree. A skewed distribution of directed edges
    would reveal an imbalance in the constructed overlay, or reveal
    vulnerable points in the system.

    \begin{description}

    \item[Betweenness centrality]\hfill\\
        Betweenness centrality is the number of times a particular node
        is found on the shortest path between two other nodes. A node
        with a big betweenness centrality may constitute both a
        vulnerable part of the graph as well as a bottleneck, as it
        might take part in a high number of event disseminations.

    \item[Topic diameter]\hfill\\
        This is the maximum number of hops between any two nodes that
        share interests, i.e.\ a measure of the diameter of a subgraph
        consisting only of nodes who registered their interest in the
        same topic. Having a low topic diameter is beneficial for
        disseminating events for topics.

    \item[Number of control messages]\hfill\\
        Some systems rely on
        control messages in order to maintain the overlay topology. For
        example in Scribe, where the multicast tree structures are
        maintained with periodic heartbeat messages. This constitutes an
        overhead both in communication and in overlay maintenance.

    \item[Clustering coefficient]\hfill\\
        This is the ratio of number of edges between neighbours of a node $n$ over
        the total number of possible edges between them. In simpler
        terms, how
        many of a nodes neighbours are connected to each other. A high
        clustering coefficient would indicate that the network has a
        higher risk of partitioning, as well as a risk of having a
        higher number of redundant message deliveries.

    \item[Partitionability]\hfill\\
        It could be useful to measure the minimal number of edges that
        need to be removed in order to partition the graph in order to
        evaluate connectivity.

    \item[Expander graph metrics]\hfill\\
        Another measure of connectivity would be expander graph metrics
        such as vertex expansion and edge expansion, which measures the
        boundary of a subgraph $S$ that is no bigger than half the total
        number of nodes in the system. The vertex boundary is the set of
        vertices outside $S$ which has at least one neighbour in $S$.
        While the edge boundary would be the set of edges with exactly
        one endpoint in $S$.

    \end{description}

    Note that some of these metrics overlap with dissemination concerns
    such as number of control messages, which could be heartbeats
    used to maintain an overlay, or control messages that are part of
    the routing protocol. Regardless, it is possible to evaluate these
    two aspect of the protocols separately.

\subsubsection{Routing metrics}
    When evaluating the routing capabilities of the protocol, several
    metrics could be considered:

    \begin{description}
    \item[Hit-ratio during churn] \hfill\\
        It is essential to understand how the different systems respond to
        churn when disseminating events. If an overlay is robust, it should
        provide a high hit-ratio in the presence of realistic churn.
        Meaning that at very high percentage of subscribers receive the
        appropriate events.

    \item[Average message delay] \hfill\\
        Counting the average number of nodes that are traversed in the
        overlay before an event reaches its target subscribers is
        helpful to understand the efficiency of the event dissemination.

    \item[Number of duplicate messages] \hfill\\
        If gossiping is used, subscribers are in danger of receiving the
        same event more than once. It would be interesting to measure
        the number of duplicates as it provides insight into the message overhead of the
        system.

    \item[Number of messages handled by a node per time unit] \hfill\\
        This metric would be one way to measure how load is distributed
        over the nodes in the system, where the time unit could be both
        seconds or cycles. Including the standard and mean deviation of
        this metric could tell whether or not the system is balanced in
        this regard.

    \end{description}

    Note that several of these metrics are irrelevant for systems such
    as SpiderCast, StaN and ElastO who do not focus on event dissemination, but
    rather the construction and maintenance of the overlay itself.  We
    will look further into which set of metrics to measure, but some
    metrics should be essential. One of these metrics is hit-ratio
    during churn, as it provides basic information regarding both the
    robustness and reliability of the dissemination overlay. Another
    central issue to consider is the increase in node degree of a
    certain node compared to the number of topics that node is
    interested in. This would reveal any scalability issues regarding
    this aspect of the protocol. As mentioned in section 3, different
    systems have different approaches to maintaining a low node degree.
    Comparing the different protocols is an interesting study in what approach
    might be most suitable for online social networks.
