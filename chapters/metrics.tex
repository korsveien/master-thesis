%TODO: Preamble

\section{Structural Overlay Properties}

\subsection{Node Degree}

It is essential to understand how the system scales with regard
to the number of topics a node is interested in. For example, if the number of
edges increases linearly with the subscription size of a node,
scalability suffers.

Also it might be interesting to measure the maximum node degree. As an
example, having a relatively low average node degree while still having
a rather high maximum node degree might reveal an unbalanced
distribution of node degree in the graph.

The two metrics above could also be separated into measuring both
in-degree and out-degree. A skewed distribution of directed edges
would reveal an imbalance in the constructed overlay, or reveal
vulnerable points in the system.

\subsection{Topic diameter}

This is the maximum number of hops between any two nodes that
share interests, i.e.\ a measure of the diameter of a subgraph
consisting only of nodes who registered their interest in the
same topic. Having a low topic diameter is beneficial for
disseminating events for topics.

\subsection{Clustering coefficient}

This is the ratio of number of edges between neighbours of a node $n$ over
the total number of possible edges between them. In simpler
terms, how
many of a nodes neighbours are connected to each other. A high
clustering coefficient would indicate that the network has a
higher risk of partitioning, as well as a risk of having a
higher number of redundant message deliveries.

\subsection{Partitionability}

It could be useful to measure the minimal number of edges that
need to be removed in order to partition the graph in order to
evaluate connectivity.

\subsection{Expander graph metrics}
Another measure of connectivity would be expander graph metrics
such as vertex expansion and edge expansion, which measures the
boundary of a subgraph $S$ that is no bigger than half the total
number of nodes in the system. The vertex boundary is the set of
vertices outside $S$ which has at least one neighbour in $S$.
While the edge boundary would be the set of edges with exactly
one endpoint in $S$.

\subsection{Centrality}

%TODO: small preamble regarding centrality

How ``important'' a node is globally. Inverse of farness, the sum of
distances to all other nodes. How long would it take to spread
information to all other nodes sequentially (how does this relate to
gossiping though?)

Betweenness centrality is the number of times a particular node
is found on the shortest path between two other nodes. A node
with a big betweenness centrality may constitute both a
vulnerable part of the graph as well as a bottleneck, as it
might take part in a high number of event disseminations.

\section{Disseminations Properties}

\subsection{Hit-ratio during churn}

It is essential to understand how the different systems respond to
churn when disseminating events. If an overlay is robust, it should
provide a high hit-ratio in the presence of realistic churn.
Meaning that at very high percentage of subscribers receive the
appropriate events.

\subsection{Average message delay}

Counting the average number of nodes that are traversed in the
overlay before an event reaches its target subscribers is
helpful to understand the efficiency of the event dissemination.

\subsection{Number of duplicate messages}

If gossiping is used, subscribers are in danger of receiving the
same event more than once. It would be interesting to measure
the number of duplicates as it provides insight into the message overhead of the
system.


\section{Communication Overhead}

\subsection{Number of control messages}

Some systems rely on control messages in order to maintain the overlay
topology. For example in Scribe, where the multicast tree structures are
maintained with periodic heartbeat messages. This constitutes an
overhead both in communication and in overlay maintenance.

\subsection{Number of messages handled by a node per time unit}

This metric would be one way to measure how load is distributed
over the nodes in the system, where the time unit could be both
seconds or cycles. Including the standard and mean deviation of
this metric could tell whether or not the system is balanced in
this regard.

