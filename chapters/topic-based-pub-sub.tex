\section{Topic-based Pub/Sub}

\subsection{ Overview }

The publish/subscribe communication paradigm is receiving an increasing
amount of attention from the research community, as it provides a
loosely coupled and scalable interaction scheme suitable for large-scale
distributed systems \cite{Eugster:2003}. It has also shown to be a useful
approach for several business applications such as financial data
dissemination \cite{tibcorv} and application integration \cite{goops}.
Topic-based pub/sub has also seen more novel applications such as
decentralised social networks. More specifically, the issue of
delivering notifications in such a network is a task especially suited
for this approach. This motivates further investigation into such
systems, comparing their performance and analysing their
characteristics and shortcomings. 

There are three main system components in the pub/sub interaction
scheme: the publishers, the subscribers and the event service. The
publishers publish events, and the subscribers subscribe for events. The
event service handles the management of subscriptions and publications,
as well as routing events to the subscribers. In topic-based pub/sub,
the published events are expressed as discrete \emph{topics}. This
variation of pub/sub benefits from the large amount of research in group
communication, as subscribing to Topic \emph{t} can be viewed as joining
Group \emph{t} \cite{Eugster:2003}. This makes topic-based pub/sub a
promising approach for efficient message dissemination in P2P
environments, and an increasing number of such systems are being
deployed \cite{Triantafillou:2009}. 

Many approaches to topic-based pub/sub has been proposed the last decade
\cites{Baehni:2004}{Castro:2002}{Chockler:2007}{Rahimian:2011}{Girdzijauskas:2010}{Matos:2010}{Wong:2008}{Zhuang:2001}.
However, there are several issues that remain in this field, as several
of the desired system characteristics are contradicting in nature
\cite{Setty:2012}. Among these challenges are ensuring reliable message
delivery, while remaining robust in the presence of churn. The following
section will discuss the desired properties of topic-based pub/sub in
more detail, and describe how balancing many of these properties
represent a design challenge.

\subsection{ Desired System Properties }
In order to provide correct and efficient delivery of notifications in a
decentralised social network, a high number of system properties are
deemed desirable \cite{Setty:2012}.  More specifically, these challenges
include: 

\begin{description}

\item[Correct delivery] \hfill\\ 
All notifications should be delivered to the
correct recipient. Both false positives and false negatives should be
avoided.

\item[High hit-ratio]\hfill\\
Notifications are delivered to a very high
percentage of subscribers in the presence of churn and all subscribers
when no churn is present. This is similar to correct delivery except for
not taking false positives into account.

\item[Fast recovery]\hfill\\
The overlay should quickly recover from a
period of churn.

\item[Low average node degree]\hfill\\
The overlay nodes should have a low
node degree as possible to achieve scalability with regards to number of
topics.

\item[Topic connectivity]\hfill\\
The routing of an event only includes the
subscribers who registered their interest for the topic (i.e.
relay-free routing).

\item[Scalability]\hfill\\
The system should scale in terms of number of
topics, number of nodes, number of topics a node is interested in and
number of nodes interested in a topic.

\item[Efficient dissemination]\hfill\\
Event dissemination should have a low
delay with little duplicate delivery, and the load of routing messages
should be distributed fairly.

\item[Low overlay maintenance cost]\hfill\\
Managing the overlay topology
should be as inexpensive as possible. Maintenance might include mending
dissemination structures such as multicast trees when nodes fail, but
also how to include joining nodes in the structure and allowing nodes to
leave gracefully.

\end{description}

As mentioned earlier, designing a system with all of these properties
presents a challenge, as several of the desired characteristics are
fundamentally at odds with each other. Maintaining a low node degree
makes it difficult to maintain \emph{topic connectivity}, while
avoiding duplicate message delivery conflicts with being robust in the
presence of churn. There is also a trade-off in robustness and
reliability depending on the approach taken to disseminating messages.
Specialised overlays that build dissemination structures such as
multicast trees provide fast and reliable message delivery with no
duplication of messages. However, they are fragile and susceptible to
churn. Epidemic dissemination on the other hand is more robust, but does
not provide full reliability as it lacks deterministic delivery
guarantees. 

There is also a trade-off between the navigability of the overlay and
the message overhead. Stanley Milgram famously demonstrated the
\emph{small-world phenomena} in \cite{milgram1967small} where he showed
that any two participants in a network was likely to be connected
through a low number of intermediaries. Taking this phenomena into
consideration has proved to be a useful approach when constructing
decentralised overlays, as they provide a highly navigable network due
to the small average shortest path length. A popular approach is to
create one or more long jump links between nodes to provide better
routing capabilities. More specifically, these links are usually created
by utilising a distance function in the name space, where the
probability of creating a link increases with the distance between them.
The subscription interest of such nodes are usually not taken into
consideration when creating such links. Consequently, the message
overhead in the system is increased as more relays are introduced in the
overlay.

Many existing systems suffer from shortcomings that originate wanting
to include a high number of desired properties described above. This
motivates further research into these systems and how they compare in
terms of promoting these desirable properties.

In the next section we will investigate the state-of-the-art of topic-based
pub/sub, describing different approaches and highlight the drawbacks and
shortcomings of existing systems.

