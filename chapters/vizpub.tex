In this chapter we describe \demo~\cite{korsveien2014vizpub}, a tool we
propose for visualizing the performance of overlay-based Pub/Sub
Systems. We presented poster and held a live demonstration of our system
at the ACM International Conference of Distributed Event Based Systems
(DEBS), held in Mumbai in May 2014, where it was awarded the price for
best poster/demo!

\section{System Overview}

To the best of our knowledge, \demo~is the first tool of its kind. The
tool is able to visualize the execution of any given distributed pub/sub
system step by step with respect to a set of performance metrics. Each
node in the system records records relevant data at selected
\emph{reporter intervals} during system execution. Our tool is then able to pull this data to a
single site, and compute various metrics at a system-wide scale. The
collected data from each interval is then collated into a single
.gexf file, which is interpreted by Gephi, which enable replay of system
execution offline.

Our tool supports two types of visualizations visualizations, the first is a
visualization of the overlay structure, and how it evolves over time.
The second type of animation is a hop-by-hop visualization of a single
publication message dissemination, where directed edges represent the
message dissemination path. We provide examples of both types of
visualizations later in this chapter.

There are several benefits to using a tool such as \demo. It enables
researchers and developers to gain a deeper insight into the overlay
structure as well as the publication process. It also has major benefits
as an educational tool, as it provides students with an visual
representation of both the structural evolution of the system, as well
as step-by-step animations of publication message disseminations. This
is useful in order to engage students, and facilitate deeper insight
into different pub/sub systems and their dissemination schemes. Such an
insight is also useful in order to identify potential weaknesses or
deployment anomalies of a given pub/sub systems. When developing the
tool, we encountered many scenarios where \demo~demonstrated its
usefulness. For example, when experimenting with PolderCast, we could
immediately verify that tree nodes were disconnected at the RINGS layer,
as seen in Figure~\ref{fig:pold_disc}.  Using our tool, we were able to
verify that this was caused by an artefact in the input workload where
the three nodes had no overlapping interest with any other node in the
system. We were then able to verify that the nodes were connected at the
CYCLON layer. We are not aware of any other tool or framework that would
allow such easy detection and validation of system behaviour.

\begin{figure}[h]
\includegraphics[width=\linewidth]{figures/disconnected-component-poldercast.pdf}
\caption{Visualization of disconnected component in the RINGS layer of PolderCast}
\label{fig:pold_disc}
\end{figure}

Another interesting use case for our tool is comparing different pub/sub systems and
protocols visually. Users may run the different systems using the same
workload, e.g.\ subscriptions and publications, and system parameters in
order to replay the execution and compare the different systems at
selected points in time. We include such comparisons in this chapter,
were we compare PolderCast and Scribe on a set of specific performance
metrics.

\section{Supported Metrics}
\label{sec:metrics}

\section{System Architecture}

\begin{figure}[h]
\centering
\includegraphics[width=\linewidth]{figures/arch}
\caption{Architecture diagram of \demo}
\label{fig:arch}
\end{figure}

The architecture of \demo~consists of three main components, all
depicted in Figure~\ref{fig:arch}: (1) \emph{Reporter}, (2)
\emph{Collector} and (3) \emph{Visualization Unit}.  The arrows seen in
Figure~\ref{fig:arch} depicts the flow of data in the architecture.
Each node in the executing system consists of a pub/sub protocol as well
as a ``Reporter''. The ``Reporter'' is the entity responsible for
providing the raw information required to compute various performance
metrics from the individual nodes participating in the overlay. This
information is pulled at regular intervals to a central site by the
``Collector'', which translates stores the information as a single
file in the .gexf format. This file is then interpreted by the
``Visualization Unit'', which consists of a single machine running
the Gephi tool. The Collector is designed to perform the collection of
data while in online mode, while the computation, aggregation and
derivation of various metrics is performed in in offline mode. The
Visualization Unit always operate in offline mode, as it waits for the
final report to be collated into a .gexf format for playback and
visualization.

The design of \demo~supports pub/sub systems that are deployed as
real distributed systems, as well as systems that are deployed in
simulation mode. This is due to the highly modular system architecture,
with strict separation of concerns, where the operation of both the
Collector and the Visualization Unit is separate from the reporting.
\demo~is designed to be a generic tool, where the only system specific
part of the architecture is the \emph{reporter interface} outlined in
Table~\ref{table:interface}. Any researcher or developer
who wants to use our framework only needs to provide an implementation
of this interface, which enables the Collector to retrieve the relevant
data from each individual node at the specified time points. These time
points are configurable, and all the information collected from the
reporter will in effect be the change of the system state from the last
time point with regards to the various performance metrics. We call
these time points \emph{reporting intervals}. The length of the intervals
is configurable, which provides the user with control over the
granularity of data collection, and thus the granularity of the
step-by-step replay of system execution performed in the Visualisation Unit. For example, if
running simulation using PeerNet, the user may determine whether or not
the reporting intervals should encompass several simulation cycles. Or,
in a real distributed pub/sub deployment scenario, the user can
determine the time delay between every reporting interval.


\subsection{Reporter}

The Reporter is responsible of providing the relevant data necessary in
order to calculate the desired performance metrics. In order to do so,
we specify a \emph{reporter interface} which is implemented at each
individual node participating in the pub/sub overlay. This interface
enables each nodes to log certain system parameters using its local
knowledge at each reporting interval. This local information is then pulled
by the Collector at the end of each reporting interval by invoking the
reporter interface. The available method calls and what data each method
returns is described in Table~\ref{table:interface}.

% ../tables/interface.tex
\begin{table}[h]
\centering
\resizebox{\columnwidth}{!}{%
\begin{tabular}{ll}
\toprule
Method Name                                       & Returns\\
\midrule
\tt long reportId()                      & The unique id of this node\\
\tt long[] reportNeighborIds()           & The unique ids of this node's neighbors\\
\tt long[] reportTopics()                & List of topic ids this node subscribes to\\
\tt long reportControlMsgsReceived()     & Number of overlay control messages received\\
\tt long reportControlMsgsSent()         & Number of overlay control messages sent\\
\tt long reportControlBytesReceived()    & Number of overlay control bytes received\\
\tt long reportControlBytesSent()        & Number of overlay control bytes sent\\
\tt PubMessage[] reportPubMsgsReceived() & Reports list of publication messages received\\
\tt PubMessage[] reportPubMsgsSent()     & Reports list of publication messages was sent\\

% \emph{reportDuplicatePubMessages(int topic, int messageId)} :  The number of duplicates received for a specific publication\\
% \hline
\end{tabular}
}%
\caption{Reporter Interface Methods}
\label{table:interface}
\end{table}



It is easy to see how the metrics mentioned in Section~\ref{sec:metrics}
can be derived from the methods listed in Table~\ref{table:interface}.
The structural properties of the overlay such as \emph{degree},
\emph{diameter} and \emph{clustering coefficient} can be derived by
reconstructing the overlay topology. This reconstruction can be achieved
through the two very first methods listed in
Table~\ref{table:interface}, namely \texttt{reportId()} and \texttt{reportNeighborIds()}.
For example, in our reporter interface
implementation for the RINGS layer in PolderCast, each node returns its
own id as well as the ids of both ring neighbors and random neighbors.
After this information is pulled, the Collector is able to derive a
graph structure where it first builds every node reported, and then draw
directed edges between these nodes based on the reported neighbor id
information. What topics each node subscribe to is also useful in order
to derive and visualize metrics such as \emph{Topic Diameter} and \emph{Subscription
    Size}. The Collector is able to pull information regarding topic
subscriptions through the \texttt{reportTopics()} interface method call.
Each node will return a set of topic ids, and the Collector is able to
use this information to attribute topics to each node, as well as edges.
In order to add topics to edges, the Collector simply iterates through
the topic id list of each node, and looks for a neighbor who share a
subscription to the same particular topic. A \emph{topic neighbor}. If a
topic neighbor of a node is found, the topic id is added as an attribute
to the edge connecting them. Applying topic attributes to nodes and
edges, provides the Visualization Unit with the ability to strip away
nodes and edges that does not belong to a particular topic, thereby
enabling calculation of topic diameter.

The dissemination properties of a given pub/sub system such as
\emph{hit ratio}, \emph{path lengths}, and number of duplicate publication
messages received can be derived by having each node provide a list of
publication messages sent and received. In order to calculate these
dissemination metrics, the publication needs to have a particular
structure. This structure is described in Table~\ref{table:structure}.
For example, in order to calculate hit-ratio for a specific topic, we
need to divide the number of subscribers of that topic who actually
received the message with the total number of topic subscribers. We
already know the which nodes subscribe to a particular topic through the
\texttt{reportTopics()} method call, and the list of publication
messages received by a node received can be retrieved through the
\texttt{reportPubMsgsReceived()} method call. Path lengths of a
message being published on a particular topic from a particular node may
be calculated in a similar fashion, as publication messages reported from
different nodes with the same id can be ordered based on their timestamp
values.

The number of duplicate publication messages received and sent by each
node is available through the \texttt{reportControlMsgsReceived()}  and
\texttt{reportControlMsgsReceived()} respectively, while the communication
overhead incurred by control messages in terms of bandwidth consumption can be
derived by the \texttt{reportControlBytesSent()} and
\texttt{reportControlBytesSent()} method calls.

% ../tables/pubmessage.tex
\begin{table}[]
\centering
\resizebox{\columnwidth}{!}{%
\begin{tabular}{ll}
\toprule
Message item           & Description\\
\midrule
\tt long  MsgId            & Unique id of the this message\\
\tt long  TopicId          & Topic id for which this message was generated\\
\tt long  SourceId         & Id of the previous hop node\\
\tt long  DestinationIds   & Ids of the next hop nodes\\
\tt long  OriginalSenderId & Node id of the message source\\
\tt long  TimeStamp        & Timestamp of the message sent/received\\
\end{tabular}
}%
\caption{Data Structure of a Publication Message}
\label{tab:structure}
\end{table}



The structure of the publication messages outlined in
Table~\ref{table:structure} also allows for visualizing the paths of
publication messages. As, mentioned, this is one of the two types of
visualizations the Collector can output as a .gexf file (where the other
type is the overlay structure). In such a visualization the Collector
will look at the topic id of the message, and only include the nodes
interested in the particular topic in the visualization. The Collector
will then iterate through the messages sent and received by each node.
By analyzing the message further, the Collector is able to create
directed edges between the nodes which represent the path of the
publication message. The edges are dynamic, i.e.\ they include a Time
Interval attribute, enabling a step-by-step animation, where edges
appear as the animation is played back in the Visualization Unit,
tracing the path of the publication. This enables researchers,
developers as well as students to analyze publication dissemination
schemes hop-by-hop.

In addition to being able to configure at a chosen reporter interval,
users of this tool may choose to only report partial information. For
example users may choose to only report structural information such as
node ids, or only dissemination specific data such as publication
messages sent and received. This flexibility is useful if only a few
aspects of system performance need analysis.

\subsection{Collector}

The Collector is the component responsible for pulling information from
the nodes at every reporting interval. It is also responsible for
aggregating and calculating certain custom metrics. By custom, we mean
any metric that is not included in the Gephi statistics component. These
metrics are usually related to dissemination and include hit-ratio,
duplicate publication messages received and path lengths. Metrics
related to overlay structure can be calculated in Gephi. These metrics
include degree, clustering coefficient, diameter and centralities. Topic
Diameter however, is a special case. In order to calculate topic
diameter, the graph needs to be filtered down to a subgraph which only
includes nodes and edges for a given topic and calculate the metrics for
every such subgraph. To do this manually using
the Gephi GUI-client would be an time consuming and error-prone task.
Therefore, the Collector  leverages the Gephi Toolkit in order to
automate this task. The collector supports what the Gephi community
refers to as \emph{static} and \emph{dynamic} metrics. This is also
referred to in literature and in~\cite{korsveien2014vizpub} as
\emph{instantaneous} and \emph{aggregated} metrics. In this thesis, we
will refer to them as static and dynamic, in order to be consistent
with the terminology used by the Gephi community. In short, static
metrics pertains to a specific point in time, while dynamic metrics is
based on historical values. The statistics component in Gephi includes
both type of metrics, but dynamic metrics only include degree and
clustering coefficient, while the Collector is able to compute dynamic
metrics for all properties such as centralities, hit-ratio and number of
control messages sent and received.

Aggregation of data is performed by serializing each individual report received
from the reporters into temporary files which are stored on disk. The
collector will then iterate through these files and output a final
report using the .gexf file format. While collection is done in online
mode, aggregation is performed in offline mode. The offline aggregation
of data prevents the collector from acting as a bottleneck. Indeed, the
collection and aggregation of data is highly decoupled from the
execution of the pub/sub protocol itself. As an alternative,
the Reporters are also able to log reports locally, and push them to
the collector at the end of pub/sub execution.

\subsection{Visualization Unit}

The Visualization Unit is a machine running the Gephi Open Graph Viz
tool. This tool is able to interpret the .gexf file generated by the
Collector and visualize the execution of the pub/sub system in question.
Gephi provides a rich GUI-experience where the user may interact with
the graph representation, apply layout algorithms, filter the graph
, execute metrics, apply color and size based on graph
properties and animate the graph evolving over time through the timeline
component. The Gephi software architecture is highly modular and
supports extensions via plugins, some of which are available in a
official plugin marketplace found at~\cite{gephimarketplace}. New
metrics, filters or database support may be implemented through such
plugins by developers and published to the marketplace free of charge.

Gephi provides many tools and components which are useful in the context
of researching and analysing pub/sub overlays:

\begin{figure}[h]
    \centering
    \includegraphics[width=\textwidth]{figures/gui_ann}
    \caption{Snapshot of Gephi GUI with several important components
        annotated}
\end{figure}

\begin{description}

\item[Node and Edge pencil tools] \hfill \\

    These two tools enable the user to create nodes and edges by
    clicking in the graph view. Edges can be undirected or directed,
    where direction is indicated with an arrow. These two tools combined
    enables building a graph by hand.

    Building such graphs can be useful in order to reason, analyze or
    learn network algorithms such as event dissemination algorithms.
    For example, the user can start with a single node which can act as
    the event source, and build the topology as the event disseminates,
    carefully following the particular algorithm in question when doing
    so. The user can also add attributes to the nodes and edges either
    through the node query tool or in the Data Laboratory component
    which also aids in visualising and understanding properties,
    drawbacks and advantages of such algorithms.

\item[Node Query Tool] \hfill \\

    With the node query tool the user is able to click on a node on the
    graph model, and a panel will appear to the top right with information
    regarding the properties and attributes of this node. Properties include
    data describing the visual properties of the node such as size, position
    and color, while attributes include the id and label and time Interval
    attributes and any additional user defined attributes. In our case, such
    user defined attributes would include topics, subscription size and
    number of control messages sent and received.

    Both the properties and attributes of the node are editable through this
    panel view. The user can select a property to change the visual
    representation of the node, or the attributes to change their value. The
    time interval attribute is interesting to edit in particular as it
    represents the points in time in which a node exists in the graph model.
    One example scenario is editing the time interval attribute for a certain
    nodes in order to see how it affects a particular metric as well as the
    overlay topology.

\item[Shortest Path Tool] \hfill \\

    With the shortest path tool selected, the user may click on two nodes on
    the graph model, and if there is a shortest path between them, this path
    will be highlighted with a color. This is useful in order reason about the
    relationship between key nodes in the graph, or to compare shortest path
    between several pairs of nodes.

\item[Heat Map Tool] \hfill \\

    The heat map tool enables the user to click on a node in the graph model
    and color its neighborhood based on the edge weight distance between
    them. More specifically, it sets the node color intensity lower for more
    distant nodes and stronger for nodes that are closer. Edge weight is a
    standard edge attribute that are by default set to 1. This means that in
    the default case, the visualization will represent the hop count
    distance from the particular node selected by the user. However, the
    edge weight can be edited by the user in order to represent other
    properties of a system. As an example, imagine setting the edge weight
    to represent network latency between two nodes. In this case, a
    neighboring node which is adjacent to the selected node would have a
    lower color intensity if the latency between them is higher than another
    neighboring node which is further away in terms of hop count.

\item[Timeline Component] \hfill \\

    The timeline component introduces an animation scheme for dynamic
    graphs. The user may choose playback parameters such as time
    interval size, step size and playback speed. The time interval will
    filter out a subgraph defined by the upper and lower bound of the
    interval. The evolution of the dynamic graph will then be animated
    by moving these bounds by the distance defined by the step
    parameter. The delay between each step is decided by the playback
    speed.

    The timeline enables the user to visually inspect the change in
    graph topology over time, as well as visualize and inspect node and
    edge attributes of the graph through both color, size and text
    labels which is able to change dynamically as part of the graph
    model animation. The timeline also enables jumping to a specific
    point in time and investigating the corresponding subgraph and its
    properties by changing the upper and lower bound of the time
    interval.

\item[Statistics Component] \hfill \\

    The metric component enables graph topology analysis by executing
    metrics on the graph. There are two types of metric algorithms in Gephi:
    static and dynamic. Static metrics are only able to execute on graph
    model representing a single point in time, while dynamic will traverse
    the time line by executing the metric iteratively across a set of time
    intervals. When executing a dynamic metric, the user is able to choose
    window size and time step. The window size is a time interval which will
    be moved by the step size defined by the user. Metrics are divided
    into \emph{static} and \emph{dynamic} metrics, where the former
    calculates a single value based on the currently defined time
    window, while the latter calculates a time series of values. When
    executing a dynamic metric, the user must define the time window
    size, and tick. The have the same functionality as step parameter
    when using the timeline component. When the metric executes,
    the time window will iterate through the entire time range of the
    simulation, calculating a static metric at each step. When finished,
    a time series is plotted and displayed for the user.

    The Statistics component include several metrics which are relevant
    to pub/sub overlays. Useful static metrics include, but are not
    limited to:

    \begin{itemize}
        \item{Degree (In/Out/Avg./Max/Min/Distr.)}
        \item{Avg. Cluster Coefficient}
        \item{Centrality (Beetweeness/Closeness/Eccentricity)}
        \item{Average Path length}
        \item{Radius}
        \item{Network Diameter}
        \item{Number of Shortest Paths}
    \end{itemize}

    Out of these, only degree and the clustering coefficient metrics have dynamic
    versions, where both calculates the average value over time. The
    average for dynamic metrics are calculated by dividing the sum of
    all node attribute values with the total number of nodes in both
    cases.

\item[Ranking Component] \hfill \\

    The ranking component is a key feature of Gephi which enables
    visualization based on node or edge attributes in form of color
    and size. When coloring nodes or edges, the ranking component
    will apply a gradient over the range of attribute values. The
    ranking component also include a result list, where the user may
    sort nodes based on the specified attribute value, which is
    useful for quickly finding the nodes with maximum value and
    minimum value. This is helpful for identifying bottlenecks in
    the system or potential load balancing issues.

    The ranking component also includes an auto apply feature, which
    supports visualising attributes dynamically while playing back the
    graph via the timeline component.

\item[Layout Component] \hfill \\

    The layout component enables the user to execute algorithms that
    calculates the position of the nodes. The user is able to adjust the
    parameters of these algorithms in order to manipulate the visual
    layout. The different algorithms emphasize different aspects of the
    topology. One example is the Force Atlas layout algorithm which
    simulates the effect of gravity on the nodes, where linked nodes
    attract each other and non-linked nodes are pushed apart. This
    particular algorithm is useful for visually detecting clusters and
    communities. Another useful algorithm is the Circular Layout
    algorithm, where nodes are positioned in a circle ordered on a
    specific attributes selectable by the user. This is useful in order
    to visualize node rankings on particular attributes.

\item[Filter Component] \hfill \\

    Filters may be applied to the graph in order to strip away nodes or
    edges on the basis of their attributes which also includes any
    calculated metrics. Filters may strip away nodes based on a value range if
    the attribute type is a number, or a regex match if the attribute is
    a string. Filters can also be combined through special operator filters
    representing set operations such as union and intersect.

    Filters are an essential mechanism in order to analyze subgraphs.
    One use example is the case of calculating topic diameters in pub/sub systems,
    where a subgraph can be filtered on a topic attribute. This
    allows executing the diameter metric on the resulting subgraph
    on the selected topic.

\item[Data Laboratory Component] \hfill \\

    The Data laboratory component enables the user to work with the node
    and edge attributes of the graph. This component provides the user
    with separate table views of node and edge attributes. Each row in
    these table represent a node or edge, and each column an attribute
    for that particular node or edge. Columns may be added or removed by
    the user. The data laboratory also provides functionality for
    manipulating columns such as merging two columns or creating new
    columns based on data from selected columns. Attribute data in
    columns that are static (i.e.\ has no lower or upper time interval
    bound associated with them) can be converted to dynamic through this
    component. Also, resizing or coloring all edges or nodes is possible
    through the laboratory by selecting all rows and right-clicking. In
    addition to this, the data laboratory also enables the user to export the
    data to file for further statistical analysis.

\end{description}

\section{Visualizations}
We implement a reporter interface both for PolderCast as well as Scribe.

\subsection{Overlay Structural}

\section{VizPub as a tool for evaluating pub/sub systems }

\section{Using Test-Driven Development}

Software Development Methodology is an active area of research
which is in part driven by the business needs of the private
sector\cite{janzen2005test}. One popular practice is so-called Test-Driven
Development (TDD). The promoters of TDD claims it increases
productivity and reduces the number of bugs and defects in the
code significantly~\cite{beck2003test}. Research
efforts performed at IBM~\cite{maximilien2003assessing} seems to
lend credibility to these claims. However, the use of TDD is not
prevalent in academia, and in~\cite{janzen2005test} they
recommend further research into the field in order to better
determine its effects.

Using TDD means writing tests before writing any code. There are
different types of test. \emph{Unit Tests} targets small,
independent pieces of code, typically methods within a single
module or component, while \emph{Integration Tests} aim to test
code across such modules and components in order to determine
how well they integrate with each other. In our work, we only
took advantage of Unit Tests where suitable using the
JUnit~\cite{junit} and Mockito~\cite{mockito} libraries.
We could also have benefited from a suite of integration tests,
as our implementation is heavily dependent on interoperating
components, as well as file and network IO\@. However, writing
these sort of tests would simply be too time consuming compared
to writing smaller unit tests.

The TDD approach to software development is best described through the
Red-Green-Refactor mantra, which is a central part of the
TDD-philosophy. It can be described through the following steps:

\begin{description}
    \item[Step 1:] Write a test that fails. (Red)
    \item[Step 2:] Make the test pass. (Green)
    \item[Step 3:] Refactor the code while making sure the test
        still passes. (Refactor)
\end{description}

In our experience this routine has been helpful when working
with our implementation code, as it enables us as developer to
refactor with confidence achieving more maintainable code and a
more thoughtful software design. Since we share our
implementation code with the research community by hosting it in
a open repository, any tool or method that helps us improve the
design and maintainability of our project is of great value to
us. Using TDD forced us to think more deeply about what
functionality to implement and how to structure and split the
problem domain into smaller function points. We believe that in
the end, following TDD where its suitable is beneficial to both
programmer productivity as well as programmer happiness. Also,
we are confident that this practice decreased the amount of
technical debt in our project, a problem we find to be commonplace in academia.
