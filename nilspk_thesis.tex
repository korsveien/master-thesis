\documentclass[UKenglish, a4paper]{ifimaster}
\usepackage[latin1]{inputenc}
\usepackage[T1]{fontenc,url}
\urlstyle{sf}
\usepackage{babel,textcomp,csquotes,ifimasterforside,varioref,graphicx}
\usepackage[backend=biber,style=numeric-comp]{biblatex}
\usepackage{pifont}
\usepackage{booktabs}
\usepackage{verbatim}
\usepackage{listings}

\lstset{%
    captionpos=b,
    tabsize=2,
    breaklines=true
}

\newcommand{\cmark}{\ding{51}}
\newcommand{\xmark}{\ding{55}}
\newcommand{\ampers}{\&}

\title{Implementation and Evaluation of Pub/Sub overlays for P2P Social Networks}
\subtitle{}
\author{Nils Peder Korsveien}

\addbibresource{bibliography.bib}

\begin{document}
\ififorside{}
\frontmatter{}
\maketitle{}

\chapter*{Abstract}
\tableofcontents{}
\listoffigures{}
\listoftables{}
\chapter*{Aknowledgements}
\mainmatter{}

\chapter{Introduction}
\section{Intended Audience}
Researchers, CS students with interests in the field.

\section{Chapter Overview}

\begin{description}
    \item[Chapter~\ref{ch:background}] \hfill \\
        Background.
    \item[Chapter~\ref{ch:protocol-descriptions}] \hfill \\
        Protocol Descriptions.
    \item[Chapter~\ref{ch:desired-system-properties-and-trade-offs}] \hfill \\
        Desired System Properties and Trade-offs.
    \item[Chapter~\ref{ch:metrics}] \hfill \\
        Metrics.
    \item[Chapter~\ref{ch:implementation}] \hfill \\
        Implementation.
    \item[Chapter~\ref{ch:evaluation}] \hfill \\
        Evaluation.
    \item[Chapter~\ref{ch:conclusion-and-future-work}] \hfill \\
        Conclusion and Future Work.
\end{description}

\section{Motivation}

The publish/subscribe communication paradigm is receiving an increasing
amount of attention from the research community, as it provides a
loosely coupled and scalable interaction scheme suitable for large-scale
distributed systems~\cite{Eugster:2003}. It has also shown to be a useful
approach for several business applications such as financial data
dissemination~\cite{tibcorv} and application integration~\cite{goops}.
Topic-based pub/sub has also seen more novel applications such as
decentralised social networks. More specifically, the issue of
delivering notifications in such a network is a task especially suited
for this approach. This motivates further investigation into such
systems, comparing their performance and analysing their
characteristics and shortcomings.

\section{Problem Statement}

    % \section{Motivation}
    % \section{Problem Statement}
    % \section{Thesis Outline}

\chapter{Background}
\label{ch:background}
\section{The Peer-to-Peer Network Architecture}
In a Peer-to-Peer (P2P) network every node acts as both client and
server. Every node contributes with its resources, including both
storage space and processing power. The execution of the system is
determined by a decentralized algorithm, which every node in the P2P
network must follow. No node has global knowledge of the entire network,
and no node acts as a single point of failure. This ensures a high
degree of scalability in terms of number of queries or amount of data
being processed, as every node is able to act as a server. Also, the P2P
network architecture is highly resilient to churn, as each node
independently needs to handle joins and leaves gracefully. This type of
self-organization is one of the main characteristics of the P2P network
architecture.

\section{Overlays}

\section{The Publish-Subscribe Communication Paradigm}

Publish-Subscribe is a fully asynchronous, loosely coupled,
highly scalable, event-based messaging pattern. There are three main
system components in the pub/sub interaction scheme: the publishers, the
subscribers and the event service. The publishers publish events, and
the subscribers subscribe for events, while the event service handles
managing both subscriptions and publications, as well as routing events
to the subscribers. The basic architecture of a typical pub/sub system
is outlined in Figure~\ref{fig:pubsubarch}.

\begin{figure}
\centering
\includegraphics[width=\textwidth]{figures/pubsubarch}
\caption{The basic architecture of a pub/sub system.}
\label{fig:pubsubarch}
\end{figure}

The event service functions as an intermediary between publishers and
subscribers. It provides a level of indirection, as well as an service
interface. Publishers are able to generate new events through the
\texttt{publish} service call. It is now the responsibility of the event
service to determine which subscribers are interested in receiving this
event, and how to route the event to them. The subscribers register
their interest through a \texttt{subscribe} service call. The event
service will then store each subscribers interest in order to
disseminate events correctly. The publishers are then able to cancel
their subscriptions through a \texttt{unsubscribe} service call. No
information is forwarded from subscribers to publishers or from
publishers to subscribers.

The pub/sub paradigm provides a higher degree of decoupling than other
traditional approaches. In general there are three types of decoupling
pub/sub system provides us with:

\begin{description}
  \item[Space decoupling] The publishers and subscribers does not need to
    know about each other.
  \item[Time decoupling] Events are delivered regardless of whether or
    not publishers and subscribers are online at the same time.
  \item[Synchronization decoupling] Neither publishers nor subscribers
    are blocked when attempting to perform their operations.
\end{description}

While many other approaches can provide the first two forms of
decoupling, the main advantage of pub/sub is its fully asynchronous nature.
Approaches such as tuple spaces or message queues cannot completely
provide this synchronous decoupling, as messages are retrieved in a
synchronous manner. This property is key to the suitability for pub/sub
in large distributed system.~\cite{Eugster:2003}

\subsection{Message Filtering in Pub/Sub}

The subscription semantics of the pub/sub paradigm plays an important
role in the performance and flexibility of the system as event messages
are routed and managed based on topic or content. There are three
distinct types of subscription schemes:

\begin{description}
  \item[Topic-based] Events are split into topics, usually represented by
      a string.
  \item[Type-based] Filters events based on the structure of the data.
      Provides type safety at compile time.
  \item[Content-based] Events are filtered based on a global
      list of universal event attributes.
\end{description}

Content-based provides better expressiveness in terms of filtering out
the relevant events. However, this comes at the cost of higher overhead
with regards to handling subscriptions. The complex filtering algorithms
limit the scalability of such systems with regards to the number of
subscriptions. Type based is similar to content-based in the sense that
the public members of the types together form a description of the
content of the event. Although this ties the implementation of the
pub/sub system closer to the programming language, it still suffers from
the same drawbacks as content-based.

Topic-based offer less expressiveness than the other two subscription
schemes, but better performance if the set of possible event properties
is limited. Also, topic-based is more suited for dissemination and
multicasting, as topics can be thought of as groups, where subscribing
to topic T can be equivalent to joining the group for that topic. This
is a common approach taken by several proposed pub/sub
systems\cite{needs citation}.

Traditionally, reliable multicasting of data through deterministic
dissemination has been the common approach. However, more recent
implementations investigate the potentials of probabilistic protocols,
which are more suited to the nature of decentralized systems and P2P.
These protocols do not guarantee full reliability, but provides a high
quantifiable \emph{probability} that events are delivered to all
subscribers.


\section{The Gephi Open Graph Viz Platform}

\begin{figure}
    \centering
    \includegraphics[width=\textwidth]{img/gephi1}
    \caption{The Gephi Tool supports
        visualization of graphs through coloring and sizing the visual
        graph
        representation. It also enables adding labels to nodes and
        edges. In
        this screenshot, Gephi is used to detect and visualize
    communities.}
\label{img:gephi1}
\end{figure}

Gephi~\cite{ICWSM09154} is an open source tool for exploring and
visualizing all kinds of networks, including dynamic and hierarchical
graphs. Described by the authors as ``photoshop for graphs'', Gephi
enables the user to interact with the graph structure, as well as
manipulate the colors and sizes of the visual graph representation in
order to display graph properties in an intuitive way. Gephi aims to
help researchers and data analysts in discovering patterns and revealing
hidden properties of the graph in question, as well as easily
discovering errors in the dataset. Gephi also provides a set of
statistical tools for measuring common metrics for Social Network
Analysis~(SNA) such as centrality, as well as metrics useful for general
graph topology analysis such as degree, path length and clustering
coefficient. Gephi is also useful in the emerging field of Dynamic
Network Analysis~(DNA)  as it supports temporal graphs, giving the user
the ability to filter the graph model according to a defined time
interval. It also support playback of the graph evolution, as well as
visualizing changes to graph data over time through size, color and text
labels which can be applied to both nodes and edges.

Gephi provides a rich GUI-experience where the user may interact with the
graph representation, apply layout algorithms, filter the graph
representation, execute metrics, apply color and size based on graph
properties and animate the graph evolving over time through the timeline
component. The Gephi software architecture is highly modular and
supports extensions via plugins, some of which are available in a
official plugin marketplace found at~\cite{gephimarketplace}. New
metrics, filters or database support may be implemented through such
plugins by developers and published to the marketplace free of charge.

Gephi provides many tools and components which are useful in the context
of researching and analysing pub/sub overlays.

\subsection{Useful functionality in Gephi}

\begin{description}

\item[Node and Edge pencil tools] \hfill \\

    These two tools enable the user to create nodes and edges by
    clicking in the graph view. Edges can be undirected or directed,
    where direction is indicated with an arrow. These two tools combined
    enables building a graph by hand.

    Building such graphs can be useful in order to reason, analyse or
    learn network algorithms, such as event dissemination algorithms. In
    this case, the user can start with a single node which can act as
    the event source, and build the topology as the event disseminates,
    carefully following the particular algorithm in question when doing
    so. The user can also add attributes to the nodes and edges either
    through the node query tool or in the Data Laboratory component
    which also aids in visualising and understanding properties,
    drawbacks and advantages of such algorithms.

\item[Node Query Tool] \hfill \\

    With the node query tool the user is able to click on a node on the
    graph model, and a panel which will display a panel with information
    regarding the properties and attributes of this node. Properties include
    data describing the visual properties of the node such as size, position
    and color. Attributes include the Id and Label and Time Interval
    attributes and any additional user defined attributes. In our case, such
    user defined attributes would include Topics, Subscription Size and
    Gossips Sent/Received.

    Both the properties and attributes of the node are editable through this
    panel view. The user can select a property to change the visual
    representation of the node, or the attributes to change their value. The
    Time Interval attribute is interesting to edit in particular as it
    represents the points in time in which a node exists in the graph model.
    On example scenario is editing the Time Interval attribute for a certain
    nodes in order to see how it affects a particular metric as well as the
    overlay topology.

\item[Shortest Path Tool] \hfill \\

    With the Shortest Path Tool selected, the user may click on two nodes on
    the graph model, and if there is a shortest path between them, this path
    will be highlighted with a color. It might be useful to reason about the
    relationship between key nodes in the graph, or to compare shortest path
    between several pairs of nodes. (more use cases?)

\item[Heat Map Tool] \hfill \\

    The heat map tool enables the user to click on a node in the graph model
    and color its neighborhood based on the edge weight distance between
    them. More specifically, it sets the node color intensity lower for more
    distant nodes and stronger for nodes that are closer. Edge weight is a
    standard edge attribute that are by default set to 1. This means that in
    the default case, the visualization will represent the hop count
    distance from the particular node selected by the user. However, the
    edge weight can be edited by the user in order to represent other
    properties of a system. As an example, imagine setting the edge weight
    to represent network latency between two nodes. In this case, a
    neighboring node which is adjacent to the selected node would have a
    lower color intensity if the latency between them is higher than another
    neighboring node which is further away in terms of hop count.

\item[Timeline Component] \hfill \\

    The timeline component introduces an animation scheme for dynamic
    graphs. The user may choose playback parameters such as time
    interval size, step size and playback speed. The time interval will
    filter out a subgraph defined by the upper and lower bound of the
    interval. The evolution of the dynamic graph will then be animated
    by moving these bounds by the distance defined by the step
    parameter. The delay between each step is decided by the playback
    speed.

    The timeline enables the user to visually inspect the change in
    graph topology over time, as well as visualize and inspect node and
    edge attributes of the graph through both color, size and text
    labels which is able to change dynamically as part of the graph
    model animation. The timeline also enables jumping to a specific
    point in time and investigating the corresponding subgraph and its
    properties by changing the upper and lower bound of the Time
    Interval.

\item[Statistics Component] \hfill \\

    The metric component enables graph topology analysis by executing
    metrics on the graph. There are two types of metric algorithms in Gephi:
    static and dynamic. Static metrics are only able to execute on graph
    model representing a single point in time, while dynamic will traverse
    the time line by executing the metric iteratively across a set of time
    intervals. When executing a dynamic metric, the user is able to choose
    window size and time step. The window size is a time interval which will
    be moved by the step size defined by the user. Metrics are divided
    into \emph{static} and \emph{dynamic} metrics, where the former
    calculates a single value based on the currently defined time
    window, while the latter calculates a time series of values. When
    executing a dynamic metric, the user must define the time window
    size, and tick. The have the same functionality as step parameter
    when using the \emph{Timeline Component}. When the metric executes,
    the time window will iterate through the entire time range of the
    simulation, calculating a static metric at each step. When finished,
    a time series is plotted and displayed for the user.

    The Statistics component include several metrics which are relevant
    to pub/sub overlays. Useful static metrics include, but are not
    limited to:

    \begin{itemize}
        \item{Degree (In/Out/Avg./Max/Min/Distr.)}
        \item{Avg. Cluster Coefficient}
        \item{Centrality (Beetweeness/Closeness/Eccentricity)}
        \item{Average Path length}
        \item{Radius}
        \item{Network Diameter}
        \item{Number of Shortest Paths}
    \end{itemize}

    Of these, only degree and the clustering coefficient metrics have dynamic
    versions, where both calculates the average value over time. The
    average for dynamic metrics are calculated by dividing the sum of
    all node attribute values with the total number of nodes in both
    cases.

    %TODO: describe exactly what sort of averages/data are calculated by
    each metric

\item[Ranking Component] \hfill \\

    The ranking component is a key feature of Gephi which enables
    visualization based on node or edge attributes in form of color
    and size. When coloring nodes or edges, the ranking component
    will apply a gradient over the range of attribute values. The
    ranking component also include a Result list, where the user may
    sort nodes based on the specified attribute value, which is
    useful for quickly finding the nodes with maximum value and
    minimum value, which might help in identifying bottlenecks in
    the system or potential load balancing issues.

    The Ranking component also includes an Auto Apply feature, which
    supports vizualising attributes dynamically while playing back the
    graph via the Timeline Component.

\item[Layout Component] \hfill \\

    The Layout component enables the user to execute algorithms that
    calculates the position of the nodes. The user is able to adjust the
    parameters of these algorithms in order to manipulate the visual
    layout. The different algorithms emphasize different aspects of the
    topology. One example is the Force Atlas layout algortihm which
    simulates the effect of gravity on the nodes where linked nodes
    attract each other, and non-linked nodes are pushed apart. This
    particular algorithm is useful for visually detect clusters and
    communities. Another useful algorithm is the Circular Layout
    algorithm, where nodes are positioned in a circle ordered on a
    specific attributes selectable by the user. This is useful in order
    to visualize node rankings on particular attributes.

\item[Filter Component] \hfill \\

    Filter may be applied to the graph in order to strip away nodes or
    edges on the basis of their attributes which also includes any
    calculated metrics. Filters may strip away based on a value range if
    the attribute type is a number, or a regex match if the attribute is
    a string. Filters can be combined through special operator filters
    representing set operations such as union and intersect.

    Filters are an essential mechanism in order to analyze subgraphs.
    One use example is the case of calculating topic diameters in pub/sub systems,
    where a subgraph can be filtered on a topic attribute. This
    allows executing the diameter metric on the resulting subgraph
    on the selected topic.

\item[Data Laboratory Component] \hfill \\

    The Data laboratory component enables the user to work with the node
    and edge attributes of the graph. This component provides the user
    with separate table views of node and edge attributes. Each row in
    these table represent a node or edge, and columns may be added or
    removed by the user. The Data Laboratory also provides functionality
    for manipulating columns such as merging two columns or creating new
    columns based on data from the existing columns. Attribute data
    in columns that are static (i.e.\ has no lower or upper time
    interval bound associated with them) can be converted to dynamic
    through this component. Also, resizing or coloring all edges or
    nodes is possible through the laboratory by selecting all rows
    and right-clicking.

    The laboratory also enables the user to export the data to file
    for further statistical analysis.
\end{description}

We consider tools such as Gephi to be a valuable addition to the field
of P2P protocol research. Visual Exploration of a dynamic network graph
is a useful approach to evaluating these protocols, as some properties
of the system are more easily spotted visually. For example, during our
implementation work, it was trivial to visually confirm that some edges
were missing from the graph, leading to the discovery of a critical bug
in the implementation code which would otherwise be difficult to spot.
It is also worth to note that the different actors involved in the Gephi
project has formed a legal entity in the form of The Gephi
Consortium~\cite{gephi-consortium} in order to assure future development
of this tool. This provides us with a certain degree of assurance that
this project is something well worth investing in, as the risk of it
being discontinued seems unlikely at this point in time.

\section{The Gephi Toolkit}

In addition to the GUI-client, the authors of Gephi also provide an API
through the Gephi Toolkit project. The toolkit packages essential
modules from the GUI-client into a standard Java library which can
be used by any stand-alone Java project by including it as a dependency.
We take advantage of this toolkit in our implementation work, where it
is mainly used to handle and store reports collected from PeerNet
simulations.

\section{The GEXF File format}

The GEXF (Graph Exchange XML Format) file format~\cite{gexf} is an
effort by the Gephi Consortium to define a standard language describing
complex network structures. Being developed by the same group of people,
the Gephi Tool is naturally fully compatible with this format, and is
able to both import and export GEXF files. This is also the case with
the Gephi Toolkit, as the module for handling such imports and exports
are included in this toolkit as well.

The GEXF file format is able to describe a graph through its nodes and
edges, as well as any data and dynamics associated with the graph. More
specifically, the file format is able to describe node, edges and their
associated attributes. Listing~\ref{lst:gexf-basic} provides an example
of a minimal static GEXF file, describing nodes, edges and attributes of
a graph.

\begin{figure}
\lstinputlisting[language=XML, label=lst:gexf-basic, frame=single]{listings/basic.gexf}
\caption{A GEXF description of a minimal static graph}
\end{figure}

\subsection{Dynamics}

One of the major advantages of this file format is its support for
dynamic functionalities.  Both nodes, edges and attributes may have a
defined time interval where they exist. These lifetime intervals are
described as ``spells'' if applied to nodes and edges, and as ``start''
and ``end'' XML-attributes if applied to node or edge attributes. The
GEXF file in Listing~\ref{lst:gexf-dynamics} shows an example of a
dynamic graph where spells are used in order to determine the lifetime
of the nodes. The start and end times are by default encoded as
doubles, however, dates are also supported, as seen in this example.

The support for dynamic graphs makes this file format an interesting
option for storing simulation data, and in our implementation work we
use this format extensively as part of our research effort.

\begin{figure}
\lstinputlisting[language=XML, caption={}, label=lst:gexf-dynamics,
frame=single] {listings/dynamics.gexf}
\caption{Example of  dynamic GEXF file using spells}
\end{figure}

\section{The PeerNet Simulator}

\subsection{Event Engine}

\subsection{Protocols}

\subsection{Observers}

\section{PolderCast}

\section{Scribe}


    % \section{The Peer-to-Peer Network Architecture}
    % \section{The Publish-Subscribe Communication Paradigm}
    % \section{Online Social Networks}
    % \section{Social Network Analysis}
    % \section{Summary}

\chapter{Protocol Descriptions}
\label{ch:protocol-descriptions}
\section{Scribe}
Structured overlay, Dissemination Trees
\section{PolderCast}
Hybrid overlay, RINGS, CYCLON, VICINITY, Random links (small-world)

\chapter{Design Challenges in Topic-Based Pub/Sub}
\label{ch:desired-system-properties-and-trade-offs}
\input{chapters/existing-approaches}
    % \section{Desired System Properties}
    % \section{Handling Trade-offs}
    % \section{Overlay Construction}
    % \section{Event Dissemination}
    % \section{Summary}

\chapter{Metrics}
\label{ch:metrics}
%TODO: Preamble

\section{Structural Overlay Properties}

\subsection{Node Degree}

It is essential to understand how the system scales with regard
to the number of topics a node is interested in. For example, if the number of
edges increases linearly with the subscription size of a node,
scalability suffers.

Also it might be interesting to measure the maximum node degree. As an
example, having a relatively low average node degree while still having
a rather high maximum node degree might reveal an unbalanced
distribution of node degree in the graph.

The two metrics above could also be separated into measuring both
in-degree and out-degree. A skewed distribution of directed edges
would reveal an imbalance in the constructed overlay, or reveal
vulnerable points in the system.

\subsection{Topic diameter}

This is the maximum number of hops between any two nodes that
share interests, i.e.\ a measure of the diameter of a subgraph
consisting only of nodes who registered their interest in the
same topic. Having a low topic diameter is beneficial for
disseminating events for topics.

\subsection{Clustering coefficient}

This is the ratio of number of edges between neighbours of a node $n$ over
the total number of possible edges between them. In simpler
terms, how
many of a nodes neighbours are connected to each other. A high
clustering coefficient would indicate that the network has a
higher risk of partitioning, as well as a risk of having a
higher number of redundant message deliveries.

\subsection{Partitionability}

It could be useful to measure the minimal number of edges that
need to be removed in order to partition the graph in order to
evaluate connectivity.

\subsection{Expander graph metrics}
Another measure of connectivity would be expander graph metrics
such as vertex expansion and edge expansion, which measures the
boundary of a subgraph $S$ that is no bigger than half the total
number of nodes in the system. The vertex boundary is the set of
vertices outside $S$ which has at least one neighbour in $S$.
While the edge boundary would be the set of edges with exactly
one endpoint in $S$.

\subsection{Centrality}

%TODO: small preamble regarding centrality

How ``important'' a node is globally. Inverse of farness, the sum of
distances to all other nodes. How long would it take to spread
information to all other nodes sequentially (how does this relate to
gossiping though?)

Betweenness centrality is the number of times a particular node
is found on the shortest path between two other nodes. A node
with a big betweenness centrality may constitute both a
vulnerable part of the graph as well as a bottleneck, as it
might take part in a high number of event disseminations.

\section{Disseminations Properties}

\subsection{Hit-ratio during churn}

It is essential to understand how the different systems respond to
churn when disseminating events. If an overlay is robust, it should
provide a high hit-ratio in the presence of realistic churn.
Meaning that at very high percentage of subscribers receive the
appropriate events.

\subsection{Average message delay}

Counting the average number of nodes that are traversed in the
overlay before an event reaches its target subscribers is
helpful to understand the efficiency of the event dissemination.

\subsection{Number of duplicate messages}

If gossiping is used, subscribers are in danger of receiving the
same event more than once. It would be interesting to measure
the number of duplicates as it provides insight into the message overhead of the
system.


\section{Communication Overhead}

\subsection{Number of control messages}

Some systems rely on control messages in order to maintain the overlay
topology. For example in Scribe, where the multicast tree structures are
maintained with periodic heartbeat messages. This constitutes an
overhead both in communication and in overlay maintenance.

\subsection{Number of messages handled by a node per time unit}

This metric would be one way to measure how load is distributed
over the nodes in the system, where the time unit could be both
seconds or cycles. Including the standard and mean deviation of
this metric could tell whether or not the system is balanced in
this regard.


% FIXME: should probably be a part of the background chapter

\chapter{Implementation}
\label{ch:implementation}
    \section{Implementation Strategy}
        \subsection{Using Test-Driven Development}

        Software Development Methodology is an active area of research
        which is in part driven by the business needs of the private
        sector\cite{janzen2005test}. One popular practice is Test-Driven
        Development (TDD). The promoters of TDD claim it increases
        productivity and reduces the number of bugs and defects in the
        code significantly~\cite{beck2003test}. Research
        efforts performed at IBM~\cite{maximilien2003assessing} seems to
        lend credibility to these claims. However, the use of TDD is not
        prevalent in academia, and in~\cite{janzen2005test} they
        recommend further research into the field in order to better
        determine its effects.

        Using TDD means writing test before writing any code. There are
        different types of test. \emph{Unit Test} targets small,
        independent pieces of code, typically methods within a single
        module or component, while \emph{Integration Tests} aim to test
        code across such modules and components in order to determine
        how well they integrate with each other. In our work, we only
        took advantage of Unit Tests where suitable using the
        JUnit~\cite{junit} and Mockito~\cite{mockito} libraries.
        However, as the system we develop depends on  high number of
        IO-operations through file reads and writes, as well as network
        communication, as suite of integration test could have been
        helpful.

        The TDD approach is best described through the
        Red-Green-Refactor mantra, which is a central part of
        TDD-philosophy:

        \begin{description}
            \item[Red] Write a test that fails.
            \item[Green] Make the test pass.
            \item[Refactor] Refactor the code while making sure the test still passes.
        \end{description}

        In our experience this has been a very helpful routine to follow
        in our implementation work, as it enables us as developer to
        refactor with confidence achieving more maintainable code and a
        more thoughtful software design. Using TDD forced us to think
        more deeply about what functionality to implement and how to
        structure and split the problem domain into smaller function
        points. We believe that in the end, following TDD where its
        suitable is beneficial to both programmer productivity and
        happiness. Also, we are confident that it decreased the level of
        technical debt in our implementation, a problem we find to be
        commonplace in academia. As we plan to host our implementation
        code in a public repository, sharing it with the research
        community, any tool or method that helps us improve the design
        and maintainability of our implementation is of great value to
        us.

        \subsection{Extending PeerNet for Gephi}
        \subsection{Aggregating data}
            Using the coordinator as a data aggregate. Aggregated data
            stored in gexf format.
    \section{Implementing Protocols in PeerNet}
        \subsection{Updating Existing Protocols}
        \subsection{Protocols}
        \subsection{Observers}
    \section{Extending PeerNet}
    \section{Implementation Challenges}
    \subsection{Implementing a workaround for memory leaks in the Gephi Toolkit}
    CollectorWorker -> GSON -> New JVM
    \subsection{Handling protocols with inconsistent views}
    Protocols remove some edges between nextCycle () and processEvent ()
    \subsection{Issues with dynamic graphs in Gephi}
        Problems handling dynamic attributes, playback issues caused by
        rounding errors as gephi uses double to define intervals.
    \subsection{GEXF file sizes}
        The GEXF format may not be scalable. However, gephi accepts
        archive files. Zipping gexf files reduced file sizes up to 93%.
    \subsection{Messages too large in PolderCast}
        It is often the case that taking something that works well
        locally and transparently distributing it leads to difficulties.

\chapter{Evaluation}
\label{ch:evaluation}
    \section{Experimental Setup}
    \section{Evaluation Workloads}
    \section{Experiments}
    \section{Evaluation Summary}

\chapter{Conclusion and Future Work}
\label{ch:conclusion-and-future-work}
    \section{Conclusion}
    \section{Future Work}

% End of part IV

\backmatter{}
\printbibliography{}
\end{document}
